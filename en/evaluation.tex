\chapter{Evaluation}\label{evaluation}
In this chapter, we will evaluate our rhyme detector. In the beginning, we will compare its performance with RhymeTagger. Then we will use it to analyze our dataset and calculate statistics about song lyrics.

\section{Performance evaluation on schemes}
To evaluate the quality of our Rhyme Detector, we will compare its rhyme scheme predictions with \gls{gold_data}. We have two annotated datasets we can use -- Chicago Rhyming Poetry Corpus (ChicagoRPC) and  the annotated test subset of our dataset (Genius). We will also compare our performance with that of RhymeTagger.

\subsection{Taggers}
We will be comparing different variants of the taggers to also evaluate the influence some factors have on the performance. The variants are following:
\begin{itemize}
	\item \textbf{RhymeTagger (ChicagoRPC)} -- the original version of RhymeTagger, as it was trained by Plecháč.
	\item \textbf{RhymeTagger (Genius$\frac{1}{3}$)} -- RhymeTagger trained on one third of our data.\footnote{Training of RhymeTagger was significantly more time consuming than training of our detector so we did not manage to train in on a larger portion.}
	\item \textbf{Rhyme Detector} -- our detector, as described in Chapter \ref{chap-rhyme-analysis}, with the default settings, trained until the EM algorithm reached convergence in 4\textsuperscript{th} iteration.
	\item \textbf{Rhyme Detector -- experiment} -- an experimental version of our detector that was not trained with the EM algorithm, but instead the matrix was created by counting the co-occurrences in the data and using their probabilities. In this experiment, we counted the co-occurrences for all the lines; in case the line did not have a rhyme within the window, we took the line immediately preceding it instead.
	\item \textbf{Rhyme Detector -- 1 iteration} -- our detector after the 1\textsuperscript{st} iteration of the EM algorithm.
	\item \textbf{Rhyme Detector -- perfect} -- our detector configured to only finding perfect rhymes -- the co-occurrence matrix is an identity matrix.
\end{itemize}

\subsection{Scores}
To evaluate the performance, we need to find an appropriate measure that could compare the gold scheme with the prediction. This task may seem easy at first, but we need to be careful because the straight-forward approach of comparing the schemes letter by letter would not work. If the prediction made an error in the beginning it would alphabetically shift the rest of the scheme and it would no longer match with the gold data.

\paragraph{Precision, Recall, F-score} Defined already in Section \ref{ml}, these metrics are perhaps the most commonly used for evaluation. However, they are not very suitable for scheme similarity. The difficult part is the definition of the classification groups (true positives, false positives, etc.) for rhyme schemes. If we assessed true positives by identical letter, we would run across the letter-shift problem described above. We could try to fix it by renaming the letters in scheme A to make it as similar as possible to scheme B. This would fail for some cases, e.g. if a larger rhyme group was split to two in scheme A, it would receive half the score than in scheme B. Adapting these scores for our problem would be very hard and we did not think of any satisfactory solution.

\paragraph{Last Index Score (LI)} To solve this problem, we propose Last Index Score (LI score). The idea is to convert the scheme from letter representation to last rhyme's index representation. This means for each line, we will use the index of the last line that rhymed with it. If the line does not have a rhyme, we will use index -1. With such representation, we can compare these indices directly, independently from scheme letters, and the proportion of matching indices will give us a score between 0 and 1. For an illustrative example, see Table \ref{LI_score_example}.


% \begin{table}[h!]
% 	\centering
% 	\begin{tabular}{c c | c c}
% 		\multicolumn{2}{c |}{Straight-forward} &
% 		\multicolumn{2}{c }{Last Index}\\
% 		\hline
% 		Gold & Prediction & Gold & Prediction\\
% 		\hline
% 		\color{OliveGreen}a		&\color{OliveGreen}a		&\color{OliveGreen}-1		&\color{OliveGreen}-1	\\
% 		\color{Bittersweet}a		&\color{Bittersweet}b		&\color{Bittersweet}0		&\color{Bittersweet}-1	\\
% 		\color{Bittersweet}b		&\color{Bittersweet}c		&\color{OliveGreen}-1		&\color{OliveGreen}-1\\
% 		\color{Bittersweet}b		&\color{Bittersweet}c		&\color{OliveGreen}2		&\color{OliveGreen}2	\\
% 		\color{Bittersweet}c		&\color{Bittersweet}d		&\color{OliveGreen}-1		&\color{OliveGreen}-1	\\
% 		\color{Bittersweet}c		&\color{Bittersweet}d		&\color{OliveGreen}4		&\color{OliveGreen}4	\\
% 		\color{Bittersweet}b		&\color{Bittersweet}c		&\color{OliveGreen}3		&\color{OliveGreen}3	\\
% 		\color{Bittersweet}b		&\color{Bittersweet}c		&\color{OliveGreen}6		&\color{OliveGreen}6	\\
% 		\midrule
% 		\multicolumn{2}{c |}{SCORE: 0.125} &
% 		\multicolumn{2}{c }{SCORE: 0.875}\\
% 	\end{tabular}
% 	\caption[Comparison of the straight-forward approach and LI score.]{Comparison of the straight-forward approach and LI score.}
% 	\label{LI_score_example}
% \end{table}
\begin{table}[h!]
	\centering
	\begin{tabular}{c cc cc}\toprule
	 line &
		\multicolumn{2}{c}{Straight-forward} &
		\multicolumn{2}{c}{Last Index}\\\cmidrule(r){2-3}\cmidrule(l){4-5}
      index    & Gold & Prediction & Gold & Prediction\\\midrule
		0 & \color{OliveGreen}a		&\color{OliveGreen}a		&\color{OliveGreen}-1		&\color{OliveGreen}-1	\\
		1 & \color{Bittersweet}a		&\color{Bittersweet}b		&\color{Bittersweet}0		&\color{Bittersweet}-1	\\
		2 & \color{Bittersweet}b		&\color{Bittersweet}c		&\color{OliveGreen}-1		&\color{OliveGreen}-1\\
		3 & \color{Bittersweet}b		&\color{Bittersweet}c		&\color{OliveGreen}2		&\color{OliveGreen}2	\\
		4 & \color{Bittersweet}c		&\color{Bittersweet}d		&\color{OliveGreen}-1		&\color{OliveGreen}-1	\\
		5 & \color{Bittersweet}c		&\color{Bittersweet}d		&\color{OliveGreen}4		&\color{OliveGreen}4	\\
		6 & \color{Bittersweet}b		&\color{Bittersweet}c		&\color{OliveGreen}3		&\color{OliveGreen}3	\\
		7 & \color{Bittersweet}b		&\color{Bittersweet}c		&\color{OliveGreen}6		&\color{OliveGreen}6	\\
		\midrule
		& \multicolumn{2}{c}{SCORE: 0.125} &
		\multicolumn{2}{c}{SCORE: 0.875}\\\bottomrule
	\end{tabular}
	\caption[Comparison of the straight-forward approach and LI score.]{Comparison of the straight-forward approach and LI score.}
	\label{LI_score_example}
\end{table}


\paragraph{ARI score} If we look at the rhyme scheme, we can notice that scheme letters can equally be represented as line cluster. All lines that share a scheme letter form one cluster (a rhyme group), and every line that does not have a rhyme is a cluster of its own. Adjusted Rand Index (ARI) Score\footnote{\url{https://en.wikipedia.org/wiki/Rand\_index\#Adjusted\_Rand\_index}} is a corrected-for-chance statistical measure of similarity between two data clusterings. We can therefore convert the schemes to use a unique letter for each non-rhyming line and use this score to evaluate the similarity of the schemes. 

\paragraph{} For these two scores, we include both micro and macro average on the dataset. Macro average is the average of all song scores, while micro average is the average of song scores weighted by the number of lines in each song.


\subsection{Comparison of the results}
We calculated the aforementioned scores for all variants of taggers and summed up the results in Tables \ref{reddy_eval} and \ref{my_eval}. 

They both performed better on the Chicago Poetry Corpus (see Table \ref{reddy_eval}). Surprisingly, our detector performed the best after 1\textsuperscript{st} iteration, but the difference is not substantial.

\begin{table}[h!]
	\centering
	\begin{tabular}{l | r r | r r}
	&	\multicolumn{2}{c |}{ARI} &
		\multicolumn{2}{c }{LI}\\
	\cline{2-5}	
	&macro &  micro  & macro  & micro \\
	\midrule
	RhymeTagger (ChicagoRPC) & \textbf{0.9463} & \textbf{0.9384} & \textbf{0.9635} & \textbf{0.9534} \\
	RhymeTagger (Genius$\frac{1}{3}$)& 0.8819 & 0.8822 & 0.9282 & 0.9202 \\
	\midrule
	Rhyme Detector & 0.9040 & 0.8915 & 0.9413 & 0.9272 \\
	Rhyme Detector -- experiment & 0.9025 & 0.8906 & 0.9406 & 0.9272 \\
	Rhyme Detector -- 1 iteration & \textbf{0.9066} &\textbf{0.8926} &\textbf{0.9426} &\textbf{0.9277} \\
	Rhyme Detector -- perfect &0.8824 &0.8732 &0.9293 &0.9149 \\
 \end{tabular}
	\caption{Evaluation of taggers on Chicago Rhyming Poetry Corpus.}
	\label{reddy_eval}
\end{table}

Surprisingly, as we can see in Table \ref{my_eval}, RhymeTagger (Genius$\frac{1}{3}$) performed worse on the \textit{Genius} dataset than the original pre-trained version. The possible cause is the nature of our data -- song lyrics have a significantly smaller percentage of rhymes than poems, so the collocations approach might not work as well for our dataset. This might also be the reason why our detector, trained only on our data, does not perform as well as RhymeTagger.

\begin{table}[h!]
	\centering
	\begin{tabular}{l | r r | r r}
		&	\multicolumn{2}{c |}{ARI} &
		\multicolumn{2}{c }{LI}\\
		\cline{2-5}	
		&macro &  micro  & macro  & micro \\
		\midrule
	RhymeTagger (ChicagoRPC) &  \textbf{0.6519} &\textbf{0.6627} &\textbf{0.8021} & \textbf{0.8139} \\
	RhymeTagger (Genius$\frac{1}{3}$)& 0.6309 &0.6443 &0.7883 &0.8040 \\
	\midrule
	Rhyme Detector & 0.6157 &0.6306 &0.7716 &0.7861 \\
	Rhyme detector -- experiment &\textbf{0.6359} &\textbf{0.6571} &\textbf{0.7866} &0.8020 \\
	Rhyme detector -- 1 iteration& 0.6096 &0.6256 &0.7682 &0.7832 \\
	Rhyme Detector -- perfect &0.6224 &0.6410 &0.7838 &\textbf{0.8030}\\
	\end{tabular}
	\caption{Evaluation of taggers on test subset of Genius.}
	\label{my_eval}
\end{table}



\section{Statistical analysis of the dataset}
Although our detector was trained on our dataset, it was unsupervised so we can still use our detector to evaluate this dataset and give us new statistical information about a large number of song lyrics. We ran rhyme detection for nearly half a million songs and summed up the results in Tables \ref{rhyme_line_stats}, \ref{rhyme_types_perc}, \ref{rhyme_group_size}, \ref{song_rating_stats}, and \ref{rhyme_stats}. In the rest of this section, we will look at them more closely and discuss the outcome that might be surprising, or the opposite, confirms the specific  characteristics of a particular genre. Extreme values are emphasized in the tables. Keep in mind, that the lyrics and their classification to genres is crowd-sourced and might be biased.
% \begin{table}[h!]
% 	\centering
% 	\begin{tabular}{l | r r r r r} 	
% 		Genre & 			Pop & 		Rap & 		Rock & 		R\&B & 		Country\\ 
% 		\midrule
% 		 Total songs& 293,679 & 99,185& 34,372& 5,125& 3,816 \\
% 		Total lines& 9,104,273 &5,661,603& 1,087,245& 225,344& 121,207 \\ 
% 		Total rhyming lines& 4,536,554& 2,849,905& 523,879& 117,862& 61,142 \\ 
% 		 Rhyming lines (\%) & 49.8\%& 50.3\%& 48.2\%& 52.3\%& 50.4\%  \\
% 		 Average lines per song & 31.001 & \textbf{57.081} & 31.632 & 43.970 & 31.763  \\
% 
% 	\end{tabular}
% 	\caption{General statistics about dataset and rhymes, per genre.} 
% 	\label{rhyme_line_stats}
% \end{table}

\begin{table}[h!]\centering
\begin{tabular}{l  r r r r r}\toprule
       &         & \multicolumn{4}{c}{Lines} \\\cmidrule(l){3-6}
\pulrad{Genre}& \pulrad{Songs}
                 & total     & avg per song  & rhyming   & (\%) \\\midrule
Pop    & 293,679 & 9,104,273 & 31.001        & 4,536,554 & 49.8 \\
Rap    &  99,185 & 5,661,603 &\textbf{57.081}& 2,849,905 & 50.3 \\
Rock   &  34,372 & 1,087,245 & 31.632        &   523,879 & 48.2 \\
R\&B   &   5,125 &   225,344 & 43.970        &   117,862 & \textbf{52.3} \\
Country&   3,816 &   121,207 & 31.763        &    61,142 & 50.4 \\\bottomrule
\end{tabular}
\caption{General statistics about dataset and rhymes, per genre.} 
\label{rhyme_line_stats}
\end{table}

In Table \ref{rhyme_line_stats}, we sum up the basic information for all genres including the portion of lines that rhyme and we can already see some interesting results. Surprisingly, the highest portion of rhyming lines is in the R\&B genre. We do not see any characteristic of this genre that could cause this. However, it is not a big difference and maybe having more examples from this genre would make it less significant. 

We can see that throughout genres typically only about half of the lines rhyme. There are more reasons that could cause this. One possibility is that rhyming in songs is not as essential as perhaps in poems and there are in fact no more rhymes to be detected. A more likely possibility is that there are more rhymes but were not detected, either because our detector did not see them or because of the imperfect formatting of our dataset. For example, in Figure \ref{web-form}, lines 4 and 5 could have a rhyme \textit{kiss you} -- \textit{miss you}, but on the 5\textsuperscript{th} line \textit{miss you} is followed by ``, babe'' what causes this rhyme to go unnoticed by our detector.


Predictably, rap has a significantly higher average number of lines per song, which confirms the fact that this genre is more talkative. What may be unexpected is that it is nearly two times more than for the other genres -- only R\&B slightly stands out but that can be explained by the fact that these two genres are known to have influenced each other throughout history.

% \begin{table}[h!]
% 	\centering
% 	\begin{tabular}{l | r r r r r} 	
% 		Genre & 			Pop & 		Rap & 		Rock & 		R\&B & 		Country\\ 
% 		\midrule
% 		Perfect masculine &	72.5& 	\textbf{58.2}& 	72.3& 	70.2& 	73.5 \\
% 		Perfect feminine &	7.9&		8.4& 		7.7& 		8.5& 		6.2 \\
% 		Perfect dactylic & 	0.7 &		0.5 & 	0.9 &		0.5& 		0.3 \\  
% 		Imperfect & 		12.0& 	\textbf{22.3} & 	12.1 & 	13.5 & 	12.2 \\
% 		Forced &  			6.9 & 	\textbf{10.6} & 	7.0 & 	7.3 &		7.8 \\
% 	\end{tabular}
% 	\caption{Percentage of different rhyme types from all rhymes in the dataset, per genre.} 
% 	\label{rhyme_types_perc}
% \end{table}

\begin{table}[h!]\centering
\begin{tabular}{l r r r r r}\toprule
                     & \multicolumn{5}{c}{Genre} \\\cmidrule{2-6}
 \pulrad{Rhyme type} & Pop  & Rap     & Rock & R\&B & Country\\\midrule
 Perfect             & 81.1 &\bf 67.1 & 80.9 & 79.2 & 80.0 \\
 \quad --- masculine & 72.5 &    58.2 & 72.3 & 70.2 & 73.5 \\
 \quad --- feminine  &  7.9 &     8.4 &  7.7 &  8.5 &  6.2 \\
 \quad --- dactylic  &  0.7 &     0.5 &  0.9 &  0.5 &  0.3 \\  
 Imperfect           & 12.0 &\bf 22.3 & 12.1 & 13.5 & 12.2 \\
 Forced              &  6.9 &\bf 10.6 &  7.0 &  7.3 &  7.8 \\\bottomrule
\end{tabular}
\caption{Percentage of different rhyme types from all rhymes in the dataset, per genre.} 
\label{rhyme_types_perc}
\end{table}


Next, Table \ref{rhyme_types_perc} shows distribution of different rhyme types. It did not come as a surprise that the most common type, by a long shot, is perfect masculine. The reasons behind this might be several -- not only has perfect match the strongest effect melodically, it is also the easiest to come up with, and makes the lyrics easy to remember. The multi-syllable perfect rhymes have a lower percentage as longer matching word pairs are rather rare. The amount of forced rhymes might be higher in reality because their detection is inherently the hardest and they might be missed more often.


Concerning rhyme types, we see that genres are generally not very different, except for rap. Rap is very unique with rhymes, its artists are known for playing with them more creatively, using \gls{internal_rhyme}s, consonance, and assonance more often. They frequently play with emphasis what can be seen as a rapid increase in imperfect rhymes. There are more forced rhymes as well and perfect rhymes are decreased as a result.

% \begin{table}[h!]
% 	\centering
% 	\begin{tabular}{l | r r r r r} 	
% 		Genre & 			Pop & 		Rap & 		Rock & 		R\&B & 		Country\\ 
% 		\midrule
% 		2-syllable rhymes & 91.1& 90.3& 91.1& 90.6 & 93.3\\
% 		5-syllable rhymes& 8.2& 9.2& 8.0& 8.9& 6.4  \\
% 		8-syllable rhymes& 0.7& 0.5& 0.9& 0.5& 0.3 \\
% 		Perfect sound match & 93.1&\textbf{ 89.4}& 93.0& 92.7& 92.2  \\
% 		Stress moved & 14.5& \textbf{28.3}& 14.5& 16.5& 14.8 \\
% 		
% 	\end{tabular}
% 	\caption{Statistics about rhyme properties in general, disregarding rhyme types, in percentage from total rhymes.} 
% 	\label{rhyme_stats}
% \end{table}
\begin{table}[h!]\centering
\begin{tabular}{l r r r r r}\toprule
                                 & \multicolumn{5}{c}{Genre} \\\cmidrule{2-6}
 \pulrad{Rhyme category}         & Pop  & Rap     & Rock & R\&B & Country\\\midrule
 1-syllable (2-component) rhymes & 91.1 &    90.3 & 91.1 & 90.6 & 93.3 \\
 2-syllable (5-component) rhymes &  8.2 &     9.2 &  8.0 &  8.9 &  6.4 \\
 3-syllable (8-component) rhymes &  0.7 &     0.5 &  0.9 &  0.5 &  0.3 \\
 Perfect sound match             & 93.1 &\bf 89.4 & 93.0 & 92.7 & 92.2 \\
 Stress moved                    & 14.5 &\bf 28.3 & 14.5 & 16.5 & 14.8 \\\bottomrule
\end{tabular}
\caption{Statistics about rhyme properties in general, disregarding rhyme types, in percentage from total rhymes.} 
\label{rhyme_stats}
\end{table}


Table \ref{rhyme_stats} is quite similar to the previous table, but by counting syllables regardless of rhyme type, and evaluating sound match and stress separately, it offers us a little bit different angle. The low percentage of multi-syllable rhymes may be caused by the fact that we only look at the components after the last stress and stress is more often on the last or penultimate syllable -- even if more syllables rhyme, only shorter rhyme will be detected.

The decreased match in sound and increased moving of stress in rap confirm the unique properties of rap we have seen previously.

A slightly increased percentage of 1-syllable rhymes in country may be noteworthy but we see no significant properties of country that could support this as a general claim.

% \begin{table}[h!]
% 	\centering
% 	\begin{tabular}{l | r r r r r} 	
% 		Genre & 			Pop & 		Rap & 		Rock & 		R\&B & 		Country\\ 
% 		\midrule
% 		Average groups per song& 6.134 &\textbf{11.484} &6.091 &8.620 &6.676  \\
% 		Average groups per 100 lines &19.787 &20.119 &19.255 &19.605 &\textbf{21.018} \\
% 		Max groups per song & 169 &224 & 81 & 48 &98\\
% 		Average group size & 2.518 &2.502 &2.502 &2.668 &\textbf{2.400} \\
% 		Max group size & 159 &98 & 68 & 42 & 24\\
% 	\end{tabular}
% 	\caption{Statistics about rhyme group size per genre.} 
% 	\label{rhyme_group_size}
% \end{table}

\begin{table}[h!]\centering
\begin{tabular}{l rr r rr}\toprule
       & \multicolumn{2}{c}{groups per song}
                        & groups per & \multicolumn{2}{c}{group size} \\\cmidrule(r){2-3}\cmidrule(l){5-6}
\pulrad{Genre}&avg& max & 100 lines&     avg &   max \\\midrule
Pop    &     6.13 & 169 &    19.79 &    2.52 &   159 \\
Rap    &\bf 11.48 & 224 &    20.12 &    2.50 &    98 \\
Rock   &     6.09 &  81 &    19.26 &    2.50 &    68 \\
R\&B   &     8.62 &  48 &    19.61 &    2.67 &    42 \\
Country&     6.68 &  98 &\bf 21.02 &\bf 2.40 &\bf 24 \\\bottomrule
\end{tabular}
\caption{Statistics about rhyme group size and counts per genre.} 
\label{rhyme_group_size}
\end{table}


Table \ref{rhyme_group_size} summarizes statistics concerning size of rhyme groups. We can observe nearly double average size for rap compared to other genres, which directly corresponds to nearly double average song length, as we have seen in Table \ref{basic_stats}. 

An interesting observation can be made about country -- average number of rhyme groups per 100 lines is slightly higher than for other genres. This corresponds with average group size being lower -- obviously country tends to contain more and smaller rhymes groups. It would be interesting to know whether this is only a property of our dataset or a property of country music in general. Although the maximum group size does not tell us any general information about the group because it may only be an outlier, it is still interesting to see that this number is again the smallest for country.

% \begin{table}[h!]
% 	\centering
% 	\begin{tabular}{l | r r r r r} 	
% 		Genre & 			Pop & 		Rap & 		Rock & 		R\&B & 		Country\\ 
% 		\midrule
% 		Average song rating& 0.432 & \textbf{0.599} &0.420 &0.520 &0.456  \\
% 		Median & \textbf{0.521} & 0.380 &0.357 &0.235 & 0.269\\
% 	\end{tabular}
% 	\caption{Song rating per genre.} 
% 	\label{song_rating_stats}
% \end{table}
\begin{table}[h!]\centering
\begin{tabular}{l r r r r r}\toprule
                    & \multicolumn{5}{c}{Genre} \\\cmidrule{2-6}
                    & Pop      & Rap      & Rock  &	R\&B  & Country\\\midrule
Average song rating &    0.432 &\bf 0.599 & 0.420 & 0.520 & 0.456 \\
Median song rating  &\bf 0.521 &    0.380 & 0.357 & 0.235 & 0.269 \\\bottomrule
\end{tabular}
\caption{Song rating per genre.} 
\label{song_rating_stats}
\end{table}

Looking at average and median song ratings in Table \ref{song_rating_stats}, we can observe two curious extremes -- rap having the highest average rating and pop with the highest median rating.
So there must be some extremely highly rated rap songs that pulled up the average.

Although we did not predict this result, it can be a sign that some artists took the importance of rhyme in rap very seriously and elaborately incorporated it densely into their lyrics.

The highest median in pop shows that many pop songs are filled with more rhymes, which can be explained by their strong tendency to be memorable. However, it seems that there are some low extremes that pulled the average rating down.
