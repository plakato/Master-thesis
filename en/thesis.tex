%%% The main file. It contains definitions of basic parameters and includes all other parts.

%% Settings for single-side (simplex) printing
% Margins: left 40mm, right 25mm, top and bottom 25mm
% (but beware, LaTeX adds 1in implicitly)
\documentclass[12pt,a4paper]{report}
\setlength\textwidth{145mm}
\setlength\textheight{247mm}
\setlength\oddsidemargin{15mm}
\setlength\evensidemargin{15mm}
\setlength\topmargin{0mm}
\setlength\headsep{0mm}
\setlength\headheight{0mm}
% \openright makes the following text appear on a right-hand page
\let\openright=\clearpage

%% Settings for two-sided (duplex) printing
% \documentclass[12pt,a4paper,twoside,openright]{report}
% \setlength\textwidth{145mm}
% \setlength\textheight{247mm}
% \setlength\oddsidemargin{14.2mm}
% \setlength\evensidemargin{0mm}
% \setlength\topmargin{0mm}
% \setlength\headsep{0mm}
% \setlength\headheight{0mm}
% \let\openright=\cleardoublepage

\usepackage[dvipsnames]{xcolor}         % pdfx imports xcolor without dvipsnames, so we need to import it first
%% Generate PDF/A-2u
\usepackage[a-2u]{pdfx}

%% Character encoding: usually latin2, cp1250 or utf8:
\usepackage[utf8]{inputenc}

%% Prefer Latin Modern fonts
\usepackage{lmodern}

%% Further useful packages (included in most LaTeX distributions)
\usepackage{amsmath}        % extensions for typesetting of math
\usepackage{amsfonts}       % math fonts
\usepackage{amsthm}         % theorems, definitions, etc.
\usepackage{bbding}         % various symbols (squares, asterisks, scissors, ...)
\usepackage{bm}             % boldface symbols (\bm)
\usepackage{graphicx}       % embedding of pictures
\usepackage{fancyvrb}       % improved verbatim environment
\usepackage{natbib}         % citation style AUTHOR (YEAR), or AUTHOR [NUMBER]
\usepackage[nottoc]{tocbibind} % makes sure that bibliography and the lists
			    % of figures/tables are included in the table
			    % of contents
\usepackage{dcolumn}        % improved alignment of table columns
\usepackage{booktabs}       % improved horizontal lines in tables
\usepackage{paralist}       % improved enumerate and itemize
\usepackage[colorinlistoftodos]{todonotes} % leave in text comments
\def\MP#1{\todo[color=green!40,inline]{#1}}

\usepackage{tipa} % use IPA characters
\usepackage[utf8]{inputenc}
\usepackage[T1]{fontenc}
\usepackage[toc, section=chapter]{glossaries}
\makeglossary

\newglossaryentry{syllable_peak}
{
	name=syllable peak,
	description={A nucleus of a syllable - either a vowel or a syllabic consonant}
}

\newglossaryentry{internal_rhyme}
{
	name={internal rhyme},
	description={A rhyme that occurs in the middle of lines of poetry, instead of at the ends of lines.}
}
\newglossaryentry{end_rhyme}
{
	name={end rhyme},
	description={Rhyme at the end of line.}
}
\newglossaryentry{gold_data}
{
	name={gold data},
	description={In data classification, it is the dataset with correct labels already assigned. It can be used for supervised learning or an evaluation of unsupervised learning.}
}
\newglossaryentry{transformer_model}
{
	name=transformer model,
	description={A deep learning model that adopts the mechanism of attention, differentially weighing the significance of each part of the input data.}
}

\newglossaryentry{rhyming_part}
{
	name=rhyming part,
	description={A part of word (or multiple words) that rhymes (is identical or similar in sound) with other word/words.}
}


\newglossaryentry{rime_riche}
{
	name=rime riche,
	description={A rhyme produced by agreement in sound not only of the last accented vowel and any succeeding sounds but also of the consonant preceding this rhyming vowel}
}


\newglossaryentry{consonant_clusters}
{
	name=consonant cluster,
	description={A sequence of syllables without a vowel}
}

\newglossaryentry{quatrain}
{
	name=quatrain,
	description={A type of stanza consisting of four lines}
}

\newglossaryentry{LSTM}
{
	name=LSTM,
	description={Long-Sort Term Memory - a type of recurrent neural network}
}

\newglossaryentry{sonnet}
{
	name=sonnet,
	description={A poetic form traditionally containing 14 lines written in iambic pentameter with rhyme scheme \textit{abab cdcd efef gg}}
}

\newglossaryentry{headless_mode}
{
	name=headless mode,
	description={A mode in which software runs on hardware without a graphic user interface, e.g. a script in terminal}
}
\usepackage{quoting} % to indent paragraphs
\usepackage{subfig}
%\usepackage[final]{changes}
\usepackage{changes}



% To properly break urls.
%\usepackage[hyphens]{url}
\quotingsetup{leftmargin=2em, rightmargin=0in, vskip=1ex}




%%% Basic information on the thesis

% Thesis title in English (exactly as in the formal assignment)
\def\ThesisTitle{Computational analysis and synthesis of song lyrics}

% Author of the thesis
\def\ThesisAuthor{Patrícia Březinová}

% Year when the thesis is submitted
\def\YearSubmitted{2021}

% Name of the department or institute, where the work was officially assigned
% (according to the Organizational Structure of MFF UK in English,
% or a full name of a department outside MFF)
\def\Department{Institute of Formal and Applied Linguistics}

% Is it a department (katedra), or an institute (ústav)?
\def\DeptType{Institute}

% Thesis supervisor: name, surname and titles
\def\Supervisor{Mgr. Martin Popel, Ph.D.}

% Supervisor's department (again according to Organizational structure of MFF)
\def\SupervisorsDepartment{Institute of Formal and Applied Linguistics}

% Study programme and specialization
\def\StudyProgramme{Computer Science}
\def\StudyBranch{Artificial Intelligence}

% An optional dedication: you can thank whomever you wish (your supervisor,
% consultant, a person who lent the software, etc.)
\def\Dedication{%
I would like to thank my supervisors Mr. Popel, and his predecessor Mr. Hajič, for their guidance, input, and a great amount of time for weekly consultations. I also want to thank Mr. Delmonte for his collaboration and numerous modifications of SPARSAR. Most of all, I want to thank my husband and family, for their support and help with my little son -- I would not be able to finish this without them.
}

% Abstract (recommended length around 80-200 words; this is not a copy of your thesis assignment!)
\def\Abstract{%
We explore a dataset of almost half a million English song lyrics through three different processes -- automatic evaluation, visualization, and generation. We create our own rhyme detector, using EM algorithm with several improvements and adjustable parameters. This may, in some cases, replace human evaluators that cannot be used, for example, after each iteration of lyrics generator to evaluate its improvement. By creating a web-page visualization of the results with interesting matrix rhyme highlighting, we make our evaluation accessible to the public. We discuss interesting genre differences discovered by applying our automatic evaluation on the entire dataset. Finally, we explore lyrics generation using state-of-the-art GPT-2.
}

% 3 to 5 keywords (recommended), each enclosed in curly braces
\def\Keywords{%
{song lyrics}, {automatic evaluation}, {rhyme detection}, {lyrics generation}, {GPT-2}
}

%% The hyperref package for clickable links in PDF and also for storing
%% metadata to PDF (including the table of contents).
%% Most settings are pre-set by the pdfx package.
\hypersetup{unicode}
\definecolor{dgreen}{rgb}{.0,.4,.0}
\hypersetup{
	pdfdisplaydoctitle, breaklinks,
	colorlinks,
	%pdfborderstyle={/S/U/W 1}, allbordercolors=dgreen,
	linkcolor=dgreen, citecolor=dgreen, filecolor=black, urlcolor=dgreen,
	%backref=page,
	pagebackref=true,
	pdfencoding=auto,
}


% Definitions of macros (see description inside)
\include{macros}

% Title page and various mandatory informational pages
\begin{document}
\include{title}

%%% A page with automatically generated table of contents of the master thesis

\tableofcontents

%%% Each chapter is kept in a separate file
\chapter*{Introduction}
\addcontentsline{toc}{chapter}{Introduction}

As artificial intelligence keeps catching up with humans, even despite numerous attempts, in artistic fields people still prefer human-made art.
For computers, it is hard to create art, and even harder to understand and analyze it.

A piece of art in everyday life of almost everyone is music. It is a complex form, where many aspects influence the audience, i.e. melody, rhythm, lyrics, performance, etc. Although we do realize they are interconnected and may affect each other, in this thesis, we will more deeply explore only one of these aspects -- song lyrics. 

We have a large crowd-sourced dataset of almost half a million \added{English} song lyrics. At first, this sounded as a good base for learning a lyrics generator. However, as we explored rhymes and automatic analysis, we realized it is a much more interesting path to pursue. Every attempt at lyrics or poetry generation that we encountered used humans for their final evaluation. This proves, and \cite{greene2010automatic} agree, that automatic evaluation of poetry is hard.

Unfortunately, there was no sufficient rhyme detector that we could use for our case. In this thesis, we will dive more deeply into the problem and create one ourselves. It will give us the ability to analyze our dataset and draw interesting conclusions about the data. 

Additionally, we will create a web-page that demonstrates detector's capabilities and visualizes rhymes in an innovative way. With this tool, we hope to give artists, authors of poems and songs, or even amateurs a new way to explore their texts.

At the end, we will focus on lyrics generation and explore current state-of-the-art
pre-trained GPT-2 model and its capabilities in this field.

This work may include some literary or technical terms that the reader is not familiar with. For their definition, please see the \hyperref[glossary-section]{``Glossary of literary and technical terms''} section at the end of this thesis.


\section*{Outline}
In Chapter \ref{chap-related-work}, we will make we will explain the literary background such as rhyme, its types, and other literary devices. We will also describe approaches and review existing tools for rhyme detection, visualization, and lyrics generation. 

Chapter \ref{data} introduces data that we will be working with, their structure and statistics, and the steps we took to pre-process them.

The most complex part of this thesis is explained in Chapter \ref{chap-rhyme-analysis}, which specifies the details of how we perform rhyme detection in song lyrics.

Chapter \ref{evaluation} evaluates our rhyme detector and shows the statistics when we run it on our dataset.

How the output of our detector is brought to life by visualization is illustrated in Chapter \ref{visualization}.

In Chapter \ref{generation}, we describe and review the results of lyrics generation experiment.

Lastly, the results are summed up in Conclusion (Chapter \ref{conclusion}), including suggestions for future work.

\listofchanges
\chapter{Related work}\label{chap-related-work}
This chapter gives a basic overview of all relevant tools and background information researched during work on this thesis. Firstly,\comment{většinou se doporučuje First, Second,... to ``ly'' je zbytečné.} it gives literary background necessary to make \added{the} reader familiar with rhyme and its different types, to know what to look for before we start detecting them. Secondly, it describes existing tools for rhyme detection and visualization, and the different approaches they took. Lastly, it explains current state-of-art tools for lyrics generation.


\section{Rhyme types and literary devices}

There are many different definitions for what a rhyme is. It is described as ``a word that has the same last sound as another word'' by Cambridge Dictionary (\cite{walter2008cambridge}) or a ``literary device, featured particularly in poetry, in which identical or similar concluding syllables in different words are repeated'' by \cite{literarydevices2020}. The definition of what a good rhyme is even changes for different languages and time periods (\cite{zhirmunsky2013introduction}). For example, full identity in sound is highly valued in French (\textit{\gls{rime_riche}}), but less valued in English (perfect rhyme requires leading consonant sounds to differ). Some authors refrain from giving an exact definition and instead leave it to reader's intuition (\cite{plechavc2018collocation}). We will define rhyme through its different types, \replaced{which}{what} will be helpful for detection later.\comment{Lze dodat odkaz na tu sekci. Také by se v této větě mělo zmínit, že ta typologie rýmů je (hlavně) pro angličtinu.}

\subsection{Basic rhyme types}
\paragraph{Perfect rhyme} (also true rhyme, or sometimes just ``rhyme'') is the most common and superior type of rhyme. It requires two conditions to be met:

\begin{itemize}
	\item last stressed vowel and all following sounds are identical
	\item immediately preceding sounds differ
\end{itemize}

It is also the only rhyme for which the definitions are consistent (for example, see \cite{bain1867manual}, \cite{vanphonological}, \cite{bergman2017litcharts}, Wikipedia\footnote{\url{https://en.wikipedia.org/wiki/Rhyme} \added{Retrieved July 19, 2021}}).
\MP{Ale třeba ve Wikipedii nevidím tu druhou podmínku. V definici Identical rhyme Wikipedie píše  ``sometimes considered to be inferior and not a perfect rhyme after all''.
Je otázka, zda to nechcete přeformulovat taky rovnou takto: podle vás je Identical rhyme podtyp Perfect Rhyme (a uvedete ho asi rovnou před Imperfect), ale dle některých autorů (konkrétní citace) ne.
}
It can be further distinguished depending on how many syllables are involved:

\begin{itemize}
	\item \textbf{Masculine} (also single, monosyllabic) -- ``the commonest kind of rhyme, between single stressed syllables at the ends of verse'' (\cite{oxforddict2008literary}). 
	Examples: 
	
	fly /\textipa{\underline{flaI}}/ -- sky /\textipa{\underline{skaI}}/
	
	before /\textipa{bi.\underline{fO:r}}/ -- explore /\textipa{Iks.\underline{plO:r}}/
	\footnote{For the examples, we are using IPA transcriptions because it is more comfortable for human readers. \added{See Appendix~\ref{ipa} for pronunciation tables.}}
	\footnote{Stressed syllables are underlined. Syllables are separated with a dot.}
	
	\item \textbf{Feminine} (also double) -- ``a rhyme on two syllables, the first stressed and the second unstressed'' (\cite{oxforddict2008literary}). Examples: 
	
	bitten /\textipa{\underline{bI}.t@n}/ -- written /\textipa{\underline{rI}.t@n}/
	
	lazy /\textipa{\underline{leI}.zi}/ -- crazy /\textipa{\underline{kreI}.zi}/
	
	\item \textbf{Dactylic} (also triple) -- ``a rhyme on three syllables, the first stressed and the others unstressed''(\cite{oxforddict2008literary}). Examples: 
	
	amorous /\textipa{\underline{æ}.m@r.@s}/ -- glamorous /\textipa{\underline{glæ}.m@r.@s}/
	
	vanity /\textipa{\underline{væ}.nI.ti}/ -- humanity /\textipa{hju:-\underline{mæ}.nI.ti})/
	
\end{itemize}

\paragraph{Imperfect rhyme} (also slant or half rhyme)  rhymes ``the stressed syllable of one word with the unstressed syllable of another word'' (\cite{bergman2017litcharts}). Examples: 

cabbage /\textipa{\underline{kæ}.bIdZ}/ -- ridge /\textipa{\underline{rIdZ}}/

painting /\textipa{\underline{peIn}.tIN}/ -- ring /\textipa{\underline{rIN}}/

\noindent In other sources, definitions differ -- for example \cite{literarydevices2020} calls this effect ``feminine rhyme''.  On the other hand, \cite{oxforddict2008literary} and \cite{britannica} use the term ``imperfect rhyme'' for end-line consonance (see definition below) and \cite{vanphonological} uses it for end-line assonance (see definition below). For the purpose of this thesis, we would like to keep rhyme types disjoint. Therefore we will require the sounds in the imperfect rhyme to be identical, except for the stress. This will differentiate it from \textit{forced rhyme} (see below).


\paragraph{Unaccented rhyme} (also weakened rhyme) ``occurs when the relevant syllable of the rhyming word is unstressed'' (\cite{britannica}). Examples: 

hammer /\textipa{\underline{hæ}.m@r}/ -- carpenter /\textipa{\underline{kA:r}.p@n.t@r}/

\noindent The difference opposed to imperfect rhyme is that here \gls{rhyming_part}s of both words are unstressed. However, for simplicity, in the scope of this thesis we will include this category under \textit{imperfect rhymes}.


\paragraph{Identical rhyme} (also \gls{rime_riche}) is ``a kind of rhyme in which the rhyming elements include matching consonants before the stressed vowel sounds.'' This includes ``rhyming of two words with the same sound and sometimes the same spelling but different meanings e.g.:

 seen /\textipa{\underline{si\added{:}n}}/ -- scene /\textipa{\underline{si:n}}/
 
 The term also covers word‐endings where the consonant preceding the stressed vowel sound is the same: 
 
 compare /\textipa{k@m.\underline{pEr}}/ -- despair /\textipa{dIs.\underline{pEr}}/.'' (\cite{oxforddict2008literary})
 
 It is generally considered not as good as perfect rhyme because it is too predictable for the listener.\footnote{\url{https://literaryterms.net/rhyme/}}
 \comment{footnote by mělo následovat za interpunkcí (tečkou). Zde jsem to už opravil.}%
 However, all rhyme detection tools as well as gold data that we will be using (annotated by professionals) include identity in perfect rhymes. To make the comparison and evaluation with our tool easier, we will do so as well.

\paragraph{Forced rhyme} (also near rhyme) ``includes words with a close but imperfect match in sound in the final syllables'' \cite{bergman2017litcharts}. Examples: 

green /\textipa{\underline{gri:n}}/ -- fiend /\textipa{\underline{fi:nd}}/

hide /\textipa{\underline{haId}}/ -- mind /\textipa{\underline{maInd}}/

\noindent This includes the case when spelling is changed in order to make the rhyme work, e.g.:

 truth /\textipa{\underline{truT}}/ -- endu'th /\textipa{en.\underline{duT}}/ (a contraction of ``endureth'')
 
 It can also refer to using unnatural word order to get the rhyming word at the end of the line (\cite{bergman2017litcharts}) but we will not make use of this interpretation in this thesis.

\subsection{Other literary devices}
This is a short overview of other literary devices that closely correlate with forced rhyme and may, according so some sources, be considered a rhyme. We will conservatively exclude these from our classification and focus solely on rhymes occurring at the end of verse.

\paragraph{Assonance} is ``repetition of stressed vowel sounds within words with different end consonants'' (\cite{britannica}). Examples:	

quite /\textipa{\underline{kwaIt}}/ -- like /\textipa{\underline{laIk}}/

free /\textipa{\underline{fri:}}/ -- breeze /\textipa{\underline{bri:z}}/

\noindent When used at the end of verse with ending consonants having a similar sound, it is equal to forced rhyme. However, the term itself defines a literary device applicable anywhere in the poem, even in the middle of the verse. Some sources classify it as rhyme, giving it various names (\cite{vanphonological}, \cite{bergman2017litcharts}, and others).

\paragraph{Consonance} is ``the recurrence or repetition of identical or similar consonants'' (\cite{britannica}). Examples: 

country /\textipa{\underline{k@n}.tri}/ -- contra /\textipa{\underline{kA:n}.tr@}/

hickory dickory dock /\textipa{\underline{hI}.k@.ri \underline{dI}.k@.ri \underline{dA:k}}/

\noindent Similarly as assonance, it applies to repetition of consonants in any part of the verse. When seen at the end of verse, it can be considered a rhyme and again, various terms are used -- perhaps the most common is ``pararhyme'' (\cite{britannica}, \cite{oxforddict2008literary}).
\newline

The last two terms may seem as more of a tool for poets than songwriters. Surprisingly, they have found their way into song lyrics and have become a standard in genres like hip hop according to \cite{vanphonological}. From the creative point of view, it is not less sophisticated rather it enriches rhyme as we know it (\cite{brogan2016poeticterms}).


Other rhyme types exist e.g. eye rhyme where ``the spellings of the rhyming elements match, but the sounds do not, e.g. love /\textipa{\underline{l@v}}/ -- prove /\textipa{\underline{pru:v}}/'' (\cite{oxforddict2008literary}). We do not consider them relevant for song lyrics or the purpose of this thesis.


\section{Rhyme detection tools}\label{rhyme_detection_tools}
We have defined what to look for, and now we will focus on how to do it. According to \cite{plechac2017presentation}, there are 4 methods for rhyme detection. We will describe each one and evaluate existing tools. For our use case, it is important that we can use the tool for automatic evaluation -- there must be a way to run it with code whether it would be an API, \deleted{or} an executable script, or as a library/module.
\MP{Library/module by měly mít definované API, takže je divné, že to uvádíte zvlášť.
Možná místo API myslíte web service.}
Another requirement is \deleted{for it} to be able to run on a block of text and generate rhyme scheme as a result. Lastly, it has to be free and preferably open-source.

\subsection{Naive rule-based approach} 
The simplest approach is to compare for identity of phonemes at the end of lines. Noticeably,  this only detects perfect rhymes\deleted[comment={Už jste psala, že identitical rhymes považujete za perfect.}]{and identity}. Nevertheless the result will seem decent because it has 100\% recall
\MP{To myslíte 100\% recall ale jen pro perfect rhymes a jen pro ty, jejichž výslovnost je ve slovníku?
Dle takovéto definice by mělo 100\% recall asi cokoliv.\\
Nemyslíte spíš 100\% precision?
Tedy vše označené jako rým, tím rýmem opravdu je?
To spíš, i když se jistě najdou řídké výjimky, jako třeba ``read'', kde ani shodný spelling nezaručuje shodnou výslovnost --  \textipa{ri:d} vs \textipa{rEd}.
Raději tedy ``almost 100\% precision''.
Pak mi ale není jasné, proč následuje \textit{and notices all ``obvious'' rhymes}
 místo \textit{i.e. notices all ``obvious'' rhymes}.
}
and notices all ``obvious'' rhymes. Another downside of this approach is its limitation to the size of the dictionary.
\MP{O žádném the dictionary dosud nebyla řeč.
Asi by se mohlo zmínit něco jako, že English orthography is highly nonphonemic, thus a pronunciation dictionary is needed for converting graphemes to phonemes.
Dále by se mohlo zmínit, že slovní přízvuky jsou nepravidelné, tedy že ten slovník musí obsahovat i přízvuky, chceme-li detekovat právě perfect rhymes (jinak bychom do toho museli přidat i imperfect).
}
There are many rhyming dictionaries (of various quality and size) to choose from but the vast majority \deleted[comment={Myslím, že se říká ``majority is'' ale ``majority of dictionaries are''. To ``of them'' mi ale nakonec přijde zbytečné.}]{of them} uses or enhances CMU dictionary \added{(CMUdict)}.\footnote{\url{http://www.speech.cs.cmu.edu/cgi-bin/cmudict}}
\subsubsection*{Pronouncing and CMU dictionary}
Pronouncing\footnote{\url{https://pypi.org/project/pronouncing/}} is a Python library providing an interface for CMU Pronouncing Dictionary. One possibility is to install CMUdict directly, search for pronunciations for both words and compare then. \textit{Pronouncing} searches the dictionary automatically for a given word and returns a list of rhyming words. However, the list is truncated and probably better suited for writer's inspiration only.


\subsection{Advanced rule-based approach}
Enhancing the naive approach with various similarity measures or other features allows us to find more subtle rhymes like \textit{imperfect} or \textit{forced}.

\subsubsection*{Rhyme Genie}
Although this tool is not free, it is the most commercial\footnote{It was included in Grammy Awards gift bag.}
\MP{Co znamená most commercial? Jak se to měří? Nemyslíte třeba most popular?
Ale to by asi bylo to samé co most used by general public.}
and perhaps the most used by general public, so it is worth mentioning. Rhyme Genie\footnote{\url{https://www.rhymegenie.com/rhyme-genie.html}} is a desktop application for Windows and MacOS that suggests 30 different rhyme types for a given word. Additionally, \replaced{it}{in} includes sayings, clichés, idioms, and a very unique feature -- adjustable rhyme similarity. However, its use case a the
\MP{přeformulovat}
reverse of what we are looking for -- rhymes are not found, only suggested.

\subsubsection*{SPARSAR}
SPARSAR (\cite{Delmonte2014}) is also a very interesting tool for poetry analysis and expressive Text-to-speech conversion. It is originally designed for a thorough examination of a very strictly structured Shakespeare's sonnets. To achieve this, it has to run analyses on many levels -- and these results can be used to analyze any poem. It looks at the poem on three levels: phonetic (pronunciation, consonant and vowel tongue position, assonance, etc.), poetic (metrical structure, rhyme schemes, acoustic length, etc.), and semantic (sentiment, metaphorically linked words, anaphora, etc.).

User can choose between a window application with graphs and diagrams or a \gls{headless_mode} with .xml output files. Its main disadvantage for our use case is that it is written in Prolog and therefore is very strict on the input format and runs only under a specific older version of Ubuntu. 

\subsubsection*{Datamuse}
Datamuse API\footnote{\url{https://www.datamuse.com/api/}} combines the advantages of \added{a} rhyming dictionary and semantic analysis. It uses CMUdict for phonetic transcription, analyzes CommonCrawl\footnote{\url{https://commoncrawl.org/}} web data repository for forced rhymes, Google Books Ngrams (\cite{weiss2015google}) for building language model, and WordNet 3.0 (\cite{pearson2005encyclopedia}) for semantic relations. Users can send complex queries, e.g. ``words that rhyme with \textit{grape} that are related to \textit{breakfast}''. Similarly to Rhyme Genie, it focuses more on rhyme suggestion rather than rhyme detection.


\subsection{Similarity (distance) based approach}
Generally, substituting arbitrary consonant in perfect rhyme does not necessary create forced rhyme -- corresponding phonemes must have similar sound\added{s}. Objective criteria to measure this similarity can be phoneme's features e.g. plosive, nasal, fricative, voiced, etc. Rhyme detectors can use these features and other specific sound rules to calculate sound distance between two words. We found no specific tool focusing on this approach, but some listed tools like Rhyme Genie incorporated it into their algorithm. 

However, not all phonetic features contribute to similar sound the same. Furthermore, speakers from different areas perceive sound differently and may have distinct boundaries for what is similar and what is not. Consequentially, there are no written rules for similarity so other researchers tried to make AI create them.
\MP{Vždyť jste nějaká taková pravidla s hierarchií našla v nějakém článku.
Nebylo to asi moc použitelné, ale bylo to written.
Radši se takovýmto snadno rozporovatelným tvrzením vyhněte.
Tu poslední větu by bylo vhodné přeformulovat/vypustit i kvůli těm other researchers...
}

\subsection{Machine learning}\label{ml}
AI is inherently very good at solving rule-based problems
\MP{Tohle je nesmysl. Vlastně nevím, jaké problems myslíte.
Jako rule-based se obvykle označují algoritmy (nikoli problémy),
které nejsou machine-learning, ale že někdo (programátor, domain expert) napsal ta pravidla
podle vlastní intuice.
}
so it is no wonder that the majority of recent tools took this approach. We will describe one tool that uses \gls{LSTM} neural network and two tools that use \added[comment={Obecně: je-li jen název, pak bez členu -- we use EM. Ale je-li za tím obecné podstatné jméno, pak se členem -- we use the EM algorithm. Jinde si to doplňte sama.}]{the} EM algorithm, with more focus towards Rhyme Tagger, as we will use this tool later for inspiration and comparison with our detector.
\MP{Čtenář zatím neví, co je Rhyme Tagger, ani že je to jeden z těch dvou tools používajících EM.
V tomto odstavci by šlo zmínit třeba něco jako, že machine learning approaches mají tu výhodu,
že se mohou pravidla rýmování naučit z dat,
a tedy se přizpůsobit různým žánrům a typům rýmů, případně i různým jazykům.
Pak byste asi měla vysvětlit, jak fungují unsupervised machine-learning metody detekce rýmů,
tedy zdůraznit, že stačí velké množství textů s rýmy, ale ty nemusejí být anotované/otagované.
Také by se asi už zde mohl zavést termín vertical collocation dle Plecháče
a nastínit ten princip -- slova (případně jejich koncové slabiky), která se často opakují na konci blízkých řádků mají vyšší pst, že budou rýmy.
Jednou z možností, jak toho využít je EM algorithm.
}

\subsubsection*{EM algorithm}
\MP{RhymeTagger taky vyuřívá EM, což ze současného textu nemusí být zřejmé.
Tato subsubsection by šla nazvat Simple EM algorithm, nebo Reddy\&Knight's EM algorithm.
Také by zde mohl být ten princip lépe vysvětlen.
Ideálně tak, abyste to Vy před rokem z toho pochopila.
Vy to máte trochu vysvětlené o kus níž u RhymeTaggeru, ale Plecháč ten základní princip převzal od Reddy\&Knight.
Ty rozdíly asi nemusíte detailně popisovat.
Taky by šlo sem jen přidat závorku (see an explanation of the EM algorithm below in the RhymeTagger description).
}
\cite{reddy2011unsupervised} proposed a language-independent model for finding rhyme schemes in poetry. They created an unsupervised model based on EM algorithm that assigns the most probable rhyme scheme for each sequence of line-final words. It achieved good results when tested on annotated English and French corpus with poetry from 15\textsuperscript{th} to 20\textsuperscript{th} century. However its big pitfall lies in the fact that it is biased towards the rhyme schemes from golden data. It has a predefined set of all rhyme schemes found in tested data and those are the only ones it chooses from. For illustration, in a 14-line stanza it can choose from 90 schemes which is only 0.00005\% of all possible options. In 29\% of cases from French corpus it has only one choice.\footnote{\url{http://versologie.cz/talks/2017basel/}}
\MP{Není zřejmé, proč je zde tato url.
Tedy není zřejmé, že je to zdroj posledních dvou (či více) vět. Lepší by bylo:
According to Plecháč [2017], ...
a takto citovat ten zdroj (ať už slajdy, nebo ten Plecháčův článek, který jsem Vám poslal).\\
Pokud si pamatuji, tak to nejsou přesné citace (na těch slajdech je to v bodech). Kdyby to byla doslovná citace z toho zdroje (ať už článku či slajdů), tak by to mělo být v uvozovkách (a kurzívou).}

\subsubsection*{RhymeTagger}
\cite{plechavc2018collocation} came \added{up} with an open-source collocation-driven \replaced{tool}{alternative} named RhymeTagger \added{as an improvement on top of citet{reddy2011unsupervised}}.

It uses the same dataset as the previous approach with addition of a larger Czech poetry corpus.\footnote{\url{https://github.com/versotym/corpusCzechVerse}}
Each line-final word is transcribed into phonetic transcription and split into two types of components -- \gls{syllable_peak} for each syllable and \gls{consonant_clusters} in between. In the \textit{expectation} step, probabilities for each component pair are calculated based on their co-occurrence in line-final words, e.g. \added{the} conditional probability of \added{the} rhyme based on peak component pair \textipa{@U}:\textipa{@U} will be very high but for consonant component pair k:r quite low. These statistics for component pairs are then used in the \textit{maximization} step to calculate the probability for line-final word pair as a combined probability of all their components (paired by means of usual association measure).
\MP{Nerozumím, co myslíte těmi usual association measure.
Čekal bych, že párování components je triviální -- každou slabiku rozdělíme na 3 komponenty (případně prázdné) a pak párujeme od konce.}
If the probability of two words is above a given threshold they are considered a rhyme. After all such pairs are classified, probabilities are iteratively recalculated in the EM cycle. 

For words that were not successfully classified with this method, there is a fallback. The author observed that some words are now pronounced differently than during the Shakespearean era\deleted{ they were written in}, therefore using current pronunciation dictionaries may ruin the original rhyme, e.g. original near /\textipa{nE:r}/ -- there /\textipa{DE:r}/ vs. contemporary near /\textipa{nI:r}/ -- there /\textipa{DE:r}/. \added{The} original pronunciation can be therefore inferred from words with similar orthography. \replaced{\citeauthor{plechavc2018collocation}}{He} calculated rhyme probability given final character trigrams, which helped achieve higher recall. Although other methods may have better precision, collocation-driven approach wins in recall as seen in Figure \ref{screenshotRT}. 

For evaluation, they used precision, recall and F-score calculated as follows:

\[PRECISION=\frac{true\ positives}{true\ positives+false\ positives}\]

\[RECALL=\frac{true\ positives}{true\ positives+false\ negatives}\]

\[F\textrm{-}SCORE=\frac{2*PRECISION*RECALL}{PRECISION+RECALL}\]

For an intuitive view, the reader can imagine precision as how many of algorithm-detected rhymes were actually rhymes, and recall as how many rhymes were discovered. F-score describes the trade-off between the two.
\MP{Souhlasím, že je zajímavé uvést zde Figure~\ref{screenshotRT} a že k tomu je vhodné zadefinovat precision/recall/f-score.
Jinak by se to ale hodilo spíš do Kapitoly 4, kde definujete vlastní scores (Last-index, ARI).
Aspoň byste zde měla na odkázat na Kapitolu 4, že tam budou evaluační metriky lépe vysvětleny.
A tam byste zase měla aspoň zmínit Precision/Recall/F-score a diskutovat,
jak se vlastně liší třeba od Last-index score a co je k čemu vhodnější.
Tím posledním si sám nejsem jist, ale přijde mi férové aspoň přiznat,
že to jsou alternativní metriky pro měření kvality detekce rýmů.}

\begin{figure}[h]\centering
	\includegraphics[scale=0.4]{../img/plechac_eval.png}
	\caption[RhymeTagger evaluation]{Evaluation of RhymeTagger on English corpus in comparison with EM algorithm and simple rule-based approach. The x axis is the year when the tested poem was written, the y axis are the evaluation scores as described above. Reproduced from \cite{plechac2017presentation}.}
	\label{screenshotRT}
\end{figure}


\subsubsection*{Deep-speare}
As a part of their \gls{sonnet} \gls{quatrain} generating model, \cite{lau2018deep} have implemented a Rhyme component that identifies and generates rhymes. It is a unidirectional forward \gls{LSTM} (\cite{hochreiter1997long}) that learns to separate rhyming word pairs from non-rhyming. They generate input by pairing one line-final word with the other three from the same quatrain. Since the rhyme scheme of a \gls{sonnet} \gls{quatrain} is always \textit{abab}, this will result in one rhyming pair and two non-rhyming. Additional non-rhyming pairs are generated with random word sampling. Then the model with margin-based loss learns the margin separating the best pair from all the others. It returns a cosine similarity score that estimates how well do two words rhyme.

To evaluate this model, authors used phoneme matching with CMUdict 
% \footnote{\url{http://www.speech.cs.cmu.edu/cgi-bin/cmudict}} \MP{to url už jste uváděla}
and the EM model from \cite{reddy2011unsupervised} trained on their own data and they were able to outperform both \replaced{according to F-score}{based on F1 score}.



\section{Visualization tools}
In the following section, we will describe existing visualization tools for poetry. Software mentioned below focuses on poems, however song lyrics can be considered just a more structurally relaxed version of a regular poem.
\MP{Asi to můžete nechat, na Matfyzu to nikdo asi řešit nebude,
ale nemyslím si, že by to platilo obecně, že písňové texty jsou more structurally relaxed než básně.
Záleží jistě na typu básní a písní.
U většiny typů písní je nutné dodržet počet slabik a přízvuky kvůli rytmu,
což neplatí o všech typech básní (i když ignorujeme volný verš).
Bylo by zajímavé tu strukturu automaticky měřit a pak vyhodnotit, je-li hypotéza o básních vs písních pravdivá.
Na to samozřejmě v této diplomce není čas.}

\subsection*{Poem Viewer}
Quite complex and comprehensive visualization tool is Poem Viewer \citep{Abdul2013}. With no need for complicated installations it is easily available for the writers as a web-based application as shown in Figure \ref{screenshotPV}. Unfortunately, at the time of writing this thesis the upload of custom text was not working. Luckily, this is still an ongoing project so this might be just a temporary issue. Nevertheless there are some default poems available to demonstrate this software's capabilities.
\begin{figure}[h]\centering
	\includegraphics[scale=0.24]{../img/ScreenshotPV.png}
	\caption{Screenshot from Poem Viewer tool -- visualizing Love by Elizabeth Barrett Browning.}\label{screenshotPV}
\end{figure}

\begin{figure}[h]\centering
	\includegraphics[scale=0.4]{../img/snapshotPV_options.pdf}
	\caption{Available options and their default mappings in Poem Viewer.}\label{screenshotPV-options}
\end{figure}

Most of the analyzed features (shown in Figure \ref{screenshotPV-options}) focus on the phonetic aspects of the poem. After phonetic transcription to IPA\added[comment={Někdo vyžaduje v angličtině čárku po všech frázích takto předsunutých před podmět (např. za každým Nevertheless). Někdo jen u některých. Zde je velmi vhodná, aby se zabránilo čtení ``IPA users''. }]{,} users can analyze consonant features, vowel length and position, stress, syllables, word classes and sentiment using color codes and markers. A second layout offers six different graphs/animations of tongue positions during each verse. Arcs are used to mark end rhyme, alliteration, assonance, consonance, their particular frequencies and repeating words.


Overall this software, although very elaborate, feels overwhelming and confusing for an inexperienced user. Moreover, it is perhaps better suited for its original use case -- a well-structured poem -- than less regular song lyrics.


\subsection*{ProseVis}
This Java desktop visualization tool by \cite{Clement2013} analyzes text through
parts-of-speech, phonemes, stress, tone, and break index. These features are extracted using OpenMary Text-to-speech system (\cite{Schroder2006}) and predictive classification. The authors believe their visualization will present the features to user in a more human readable form (\cite{prosevis2017sourceforge}).

\begin{figure}[h]\centering
	\includegraphics[scale=0.24]{../img/prosevis.png}
	\caption[Comparison of two poems in ProseVis]{Comparison of two poems in ProseVis. Reproduced from \cite{prosevis2017sourceforge}.}\label{screenshotProsevis}
\end{figure}

\subsection*{Poemage and RhymeDesign}
Poemage (\cite{McCurdy2015poemage}) and RhymeDesign (\cite{McCurdy2015}) are both open-source applications with focus on analysis of sonic devices and sonic topology in poetry. Poemage\footnote{\url{http://www.sci.utah.edu/~nmccurdy/Poemage/}} focuses on complex structures of words connected through sonic or linguistic resemblance across the space of the poem. It is available for MacOS or Windows with a web version currently under development. In MacOS application RhymeDesign -- which also provides the backend for Poemage -- users can enter their poem and query for one of the default rhyme types or choose a custom rhyme type.

\begin{figure}[h]\centering
	\includegraphics[scale=0.24]{../img/poemage.pdf}
	\caption{An example analysis in Poemage.}\label{screenshotPoemage}
\end{figure}

\subsection*{Ambiances}
This software is unique in the fact that the analysis is integrated in the process of writing. As described in the paper \cite{Meneses2015}, writers input the poem, receive a visualization and can control this visualization with body and hand gestures which in turn influence the poem. By such interconnection the authors aim to make Ambiances a part of the writing process and give it a chance to influence the final result. However, the actual software does not seem to be publicly available.


\section{Generation tools}\label{generation_tools}

Generation of text, especially artistic, is a very challenging task for a machine. As we observed the outputs of exiting tools, we found that the most common flaws include lack of creativity, unnatural and frequent change of the subject, and incoherence. Generation of song lyrics is basically a more specified
\MP{Nemyslíte specific?}
branch of text generation. Similarly, it can be distinguished into two main \added{types of} methods -- rule-based and machine learning.
\MP{similarly to what? Asi Similarly to rhyme detection.}

\subsection{Rule-based generating}
Rule-based \replaced[comment={Model značí něco natrénovaného pomocí machine learning, tedy ne rule-based.}]{tools}{models} are inherently very complex and the output is often limited to certain structure or topic. They usually require the user to input starting configuration, whether it is genre, topic, time period, amount and type of rhymes, phrases, etc. They tend to be less creative but better at rhyming and following the form and structure. To achieve that, they may use rhyming dictionaries (see Section \ref{rhyme_detection_tools}). Simpler ones just use a large set of pre-written templates, e.g. MasterPiece Generator.\footnote{\url{https://www.song-lyrics-generator.org.uk/}} In this thesis, we will focus on generation using artificial intelligence (AI).
\MP{Obecně se AI považuje za nadmnožinu machine learning (což je nadmnožina deep learning, což se velmi překrývá s artificial neural networks).
Do AI tradičně spadají i rule-based metody.
Buď vysvětlete čtenáři, že pod AI rozumíte jen moderní machine-learning-based AI,
 nebo změňte tuto větu a následující podsekci přejmenujte.
Chtělo by to ale ověřit, že všechny uvedené online generátory jsou opravdu machine-learning based.
Tipuji, že některé budou spíš pravidlové.
Pokud to o nich víte, uveďte je v této podsekci.
Samozřejmě mohou existovat metody hybridní využívající jak ručně psaná pravidla, tak strojové učení.
Jedním z příkladů je nakonec Váš detektor.
}

\subsection{Generating using AI}
The go-to approach for generating song lyrics is certainly AI in its many forms.
\MP{go-to mi jednak nepřijde vhodné pro odborný styl, jednak to vnímám v podstatě jako best,
což je sporné tvrzení -- pokud do AI nezahrnujeme rule-based, tak třeba nějaké roky vylepšované rule-based řešení může být nejlepší.
Pokud zahrnujeme, tak je otázka, zda vůbec existují nějaké plně automatické non-AI metody
a zda má ta věta vůbec nějakou výpovědní hodnotu.
}
It can be proved by the large number of AI lyrics generators created by enthusiasts, e.g. \textit{These Lyrics Do Not Exist}\footnote{\url{https://theselyricsdonotexist.com/}}, \textit{Freshbots Lyrics Generator}\footnote{\url{https://www.freshbots.org/lyrics-generator}}, \textit{Random Lyrics Generator}\footnote{\url{http://www.anticulture.net/RandomLyrics.php}}, \textit{DeepBeat -- Rap Lyrics Generating AI}\footnote{\url{https://deepbeat.org/}}, \textit{BoredHumans Generator}\footnote{\url{https://boredhumans.com/lyrics_generator.php}}, \textit{RapPad Lyrics Generator}\footnote{\url{https://www.rappad.co/songs-about/}}, and many more. We will further describe Deep-speare (mentioned earlier in Section \ref{ml}) and current state-of-art GPT-2 and GPT-3.
\MP{Jednak state-of-\textbf{the}-art.
Jednak by mělo být jasné, v kterém ohledu jsou state-of-the-art,
 tedy nejspíš v nějaké měřitelné evaluaci.
U benchmarků typu GLUE a SuperGLUE už jsou myslím i lepší modely,
 ale není to dávno, co bylo na prvních příčkách GPT, takže tam by to pojmenování šlo použít.
Zde nás ale spíš zajímá generování poezie a v tom nevím o žádném benchmarku a ustáleném způsobu evaluace.
Možná by šlo něco jako ``and GPT models, which are considered state of the art in text generation.''
}

\subsubsection*{Deep-speare}
Deep-speare (\cite{lau2018deep}) is a joint neural network architecture that generates only a specific type of poems with strict form and meter -- \gls{sonnet} \gls{quatrain}s. It consists of three models: language model generates one word at a time, pentameter model samples meter-conforming sentences, rhyme model enforces rhyme, and they are all trained together in multi-task learning setting. They present very good results -- generated poems are mostly indistinguishable from human-written ones, apart from expert evaluation, where they report lack of emotion and worse readability.

\subsubsection*{GPT-2}
Generative pre-trained Transformer version 2 (GPT-2) (\cite{radford2019gpt2}) is an unsupervised \gls{transformer_model} capable of various text-processing and generating tasks such as answering questions, translating, summarizing, writing coherent paragraphs, etc. It was created by AI-based research laboratory named OpenAI.

This model was trained on and evaluated against WebText, a dataset consisting of the text contents of 45 million links on sites like Google, Blogspot, GitHub, NYTimes, BBC, eBay, etc. It offers 4 models of different sizes increasing in the number of parameters: 124 million (small), 355 million (medium), 774 million (large), and 1.5 billion (XL) parameter models.

Although this model was only trained for the general task of predicting the next word, given all of the previous words within some text, it can be further fine-tuned for a more specific task to suit user's needs. However, even without fine-tuning, it can quickly adapt to the style of the input and continue in the same manner. One example of lyrics generated by GPT-2 is \textit{Keywords To Lyrics}\footnote{\url{https://lyrics.mathigatti.com/}}.
\MP{Jenže zrovna toto je fine-tunované na písničkách, viz https://github.com/mathigatti/pop-lyrics-dataset
Stačí tedy změnit pořadí těch posledních dvou vět.
Po tom ``continue in the same manner'' by ještě měl následovat odkaz na Vaše experimenty v Kapitole 6.}

\subsubsection*{GPT-3}
During writing of this thesis, an even larger model GPT-3 (\cite{brown2020gpt3}) with 175 billion parameters was officially introduced. It is considered the largest artificial neural network created to date.
\MP{Radši It was considered... in May 2020. Ono se to rychle mění.
Zaznamenal jsem třeba Google Switch Transformer 6krát větší než GPT-3, ale jsou i další:
https://towardsdatascience.com/top-5-gpt-3-successors-you-should-know-in-2021-42ffe94cbbf
}
It was trained on five different corpora: Common Crawl, WebText2, Books1, Books2 and Wikipedia. The architecture is the same as in GPT-2, only number of layers and other parameters increased. It is capable of writing articles indistinguishable from human-written ones,
\MP{Zde by se velmi hodila citace.}
even produce functional JavaScript code for natural-language formulated task.

Realizing the power of this tool, the authors did not want to make it available to broad public, fearing it might be misused with bad intentions. Instead they created a form to sign up for access, reviewed individual requests, granting access only to a small portion of them.



\paragraph{}Originally, this thesis intended to focus more on lyrics generation. However, as our research of works in this field shows, there is an abundance of tools for \added{that} purpose. Users can just choose a tool that fits their needs. Creating one tool that would be better or more versatile would either require more computing power or literary knowledge, and is above the scope of a master thesis. Therefore, we shifted our main focus to automatic rhyme detection and evaluation, which seems to be far less explored and more interesting topic.



\chapter{Data}\label{data}

A crucial part of every analysis are the data. To be able to conduct an analysis with results that can reasonably represent the domain, we need to have enough of them -- the more the better. 

Our dataset consist of 658,460 song lyrics scraped from the crowd-sourced website 
Genius\footnote{\url{https://genius.com/}}. Sadly, the original author of the dataset is unknown, it has been passed on to us by a colleague as a potentially interesting source for research. However, all song lyrics are publicly available on the Genius website and can be linked with the corresponding item of the dataset via the \textit{url} attribute.

We apologize for any strong language that may be used in song lyrics or their excerpts in this thesis. Due to its various forms and the size of the dataset, it would be extremely difficult to remove them, and because they occur in pop culture naturally, we chose to portray them faithfully.

\section{Preprocessing}

In most areas it is very hard to find a dataset of good quality and large quantity. Usually at least one of the two suffers. It is not any different with our data -- although the dataset is large, the contents were created by ordinary people and intended for human readers so they are not well suited for automated processing. It is necessary to look closely at the data, remove faulty or redundant items, and clean the rest with preprocessing.

To assess what are the problems in the data and how to address them, we created a very small dataset of only about 10 songs which we cleaned manually. To select these songs, we looked at about 100 random songs and chose the ones that contained the most common faults. We also tried to contain a broad spectrum of errors by focusing on the diversity in the selected dataset. We then iteratively implemented an automated solution for each type of data corruption, comparing the automatically and manually cleaned data, until they matched. We also extracted statistical information that further showed the weak points that needed addressing.

We received the dataset in JSON format, with each song as a separate item, each containing following features:

\begin{itemize}
	\item \textit{title} -- the name of the song
	\item \textit{lyrics} -- the text of the song's lyrics
	\item \textit{album} -- song's album (or null)
	\item \textit{genre} -- one of the following: rap, pop, rock, r-b, country
	\item \textit{artist} -- song's performer
	\item \textit{url} -- the url of the lyrics page on Genius website
	\item \textit{year} -- the year the song was produced
	\item \textit{is\_music} -- boolean flag distinguishing song lyrics from other texts
	\item other song details: \textit{producer, featured artist, recording location, charts, writer, samples, sampled in, has featured video, has featured annotation}
	\item other website specific information: \textit{rg artist id, rg type, rg tag id, rg song id, rg album id, rg created, has verified callout}	
\end{itemize}

The features from the last two points were \textit{null} for all or most of the items. That, and the fact that they do not give us much more information that would contribute to the lyrics analysis, made them useless and so we decided not to keep them. We removed all songs for which the attribute \textit{is\_music} was False, indicating it was not a song (often a poem or prose) or they did not contain lyrics -- 34,259 songs in total. We further removed one song with invalid (incomplete) JSON. After comparing the lyrics to each other, we found and removed 32,551 duplicates.

Upon further inspection, we found out that our dataset also contains lyrics in different languages. We used the neural network model for language identification by Google (CLD3)\footnote{\url{https://github.com/google/cld3}} for the classification. It showed that our dataset contains exactly 100 different languages, most of them represented only marginally. Since other languages did not have enough data to support a good analysis and implementing them would be above the scope of this thesis, we kept only English lyrics. We further removed 832 items with language detection errors. All of them were under 10 lines long so they would not be a valuable addition anyway. At this point our dataset contained 438,037 items.

\begin{figure}[!htb]
	\minipage{0.49\textwidth}
	\includegraphics[width=\linewidth]{../img/histogram_song_length_in_char.png}
	\caption{Histogram of number of characters in songs of our dataset.}\label{hist_chars}
	\endminipage\hfill
	\minipage{0.49\textwidth}
	\includegraphics[width=\linewidth]{../img/histogram_song_length_in_words.png}
	\caption{Histogram of number of words in songs of our dataset.}\label{hist_words}
	\endminipage\hfill
	
\end{figure}
\begin{figure}[!h]\centering
	\minipage{0.49\textwidth}
	\includegraphics[width=\linewidth]{../img/histogram_song_length_in_lines.png}
	\caption{Histogram of number of lines in songs of our dataset.}\label{hist_lines}
	\endminipage\hfill
\end{figure}

To learn more about our data, we created histograms with song's length in characters, words, and lines (see Figures \ref{hist_chars}, \ref{hist_words}, \ref{hist_lines}). Knowing the common issues of extreme values, we manually examined a few of the shortest and a few of the longest items. Confirming our expectations, we found out that they were not valid songs either. The long ones were usually book excerpts or rap improvisation battles, while the short ones were often links to advertisements or motivational quotes. We removed 14 long and 1,838 short lyrics. Although it may not seem as much, it could have strong negative influence mainly during the generation phase. In Figures \ref{hist_chars}, \ref{hist_words}, and \ref{hist_lines}, red lines mark the borders for removal -- only the items in between them were kept.


\section{Structure of the data}

This section gives statistical information about the dataset after preprocessing. Table \ref{basic_stats} sums up basic statistics about the data overall and for each genre specifically. Pie chart in Figure \ref{piechart_genres} shows the portions of the data belonging to each genre. All the attributes are listed in Table \ref{stats_nonempty_values}. For some songs, not all attributes are available. Number of items for with non-empty values is given as well.
\begin{table}[h!]
	\centering
	\begin{tabular}{c | r r r} 
		Genre & Songs & Avg. lines per song & Avg. words per line \\ 
		\midrule \midrule
		Pop & 293,679& 36.73 & 5.50\\
		Rap & 99,189 & 64.32 & 6.88 \\
		Rock & 34,372 & 38.73 & 5.30 \\
		R\&B & 5,126 & 52.82 & 5.51 \\
		Country & 3,819 & 38.43 & 5.85 \\
		\midrule
		Total & 436,185 & 43.36 & 5.96 \\
	\end{tabular}
	\caption{Basic statistics about the dataset.}
	\label{basic_stats}
\end{table}



\begin{figure}[h]\centering
	\includegraphics[scale=0.25]{../img/piechart_genres.png}
	\caption{Distribution of genres in the dataset.}\label{piechart_genres}
\end{figure}



\begin{table}[h!]
	\centering
	\begin{tabular}{| c | r |} 
		\hline
		Attribute & Non-empty values \\ [0.5ex] 
		\hline
		lyrics & 436,185 \\
		title & 436,179 \\
		album & 112,060 \\
		genre & 436,185 \\ 
		artist & 436,184 \\ 
		url & 436,185 \\
		year & 96,491 \\ 
		lang & 436,185 \\
		id & 436,185 \\
		word\_count & 436,185 \\
		\hline
	\end{tabular}
	\caption{Attributes and their counts of non-empty values.}
	\label{stats_nonempty_values}
\end{table}

\section{Annotated subset}
From the dataset described above, we separated a subset of 50 songs, 10 song for each genre, and annotated them with rhyme schemes ourselves. We then separated them into a development and test set, so that both groups would have an approximately equal number of non-empty lines per genre (plus or minus 2 lines). We reserved the development set for iterative testing and improvement of our algorithm. The test set was used as an additional checkpoint for final evaluation, as you can read in Chapter \ref{evaluation}.

\section{Chicago Rhyming Poetry Corpus}
For an additional dataset for evaluation, we included one from an outside source -- the Chicago Rhyming Poetry Corpus\footnote{\url{https://github.com/sravanareddy/rhymedata}}, annotated with rhyme schemes by professionals. It contains 1,321 poems from 32 authors written in 14\textsuperscript{th} to 19\textsuperscript{th} century. It has a total of 93,045 lines with an average of 70.44 lines per song. It is the same dataset that was used for training and evaluation of Rhyme Tagger.





\chapter{Rhyme detection}\label{chap-rhyme-analysis}
In this Chapter, we will first consider using available tools for rhyme detection. After not finding the right tool we will reconsider to make the detection ourselves.

Detecting rhymes may seem like a simple task at first, but looking into details one discovers many problems that need to be addressed. As we have seen in Chapter \ref{chap-related-work}, it is not a well-defined task so we need to set the requirements.

Then we progress through individual steps needed to find the rhymes and design a rating that would song's rhymes:


\begin{enumerate}
	\item phonetic transcription with additional data preprocessing	
	\item syllabification and extraction of phonemes after last stressed syllable
	\item comparison of two lines and calculating their rhyme rating  
	\item finding all rhymes and assigning scheme
	\item calculating song rating
\end{enumerate}

\section{Using available tools for detection}
The simplest approach would be to use one of the tools described in section \ref{rhyme_detection_tools}. We want a detector that is free, strong (detects more than perfect rhymes), and offers headless mode (we could run it automatically from code). Ruling out unsuitable tools, we are left with Rhyme Tagger and SPARSAR. Rhyme Tagger is easy-to-use but only outputs rhyme scheme. To be able to automatically evaluate the rhyming quality of a song, we would need more information like stress or rhyme type.

\subsection*{SPARSAR}
SPARSAR, on the other hand, has a very rich and detailed output. Although it is lacking documentation, the \textit{xml} output format is quite descriptive to understand what most of the values represent. It seemed promising so we attempted to pursue this path.

To bridge the outdated system requirements, we contacted the authors for a newer build (for Ubuntu 19.1). They were very helpful and soon we were able to run it on our computer. Nevertheless, we encountered several issues, mostly stemming from the fact that SPARSAR was written in Prolog. Firstly, the xml output was difficult to parse automatically, as the values were written in Prolog syntax, using the same delimiter for different levels of separation. Secondly, SPARSAR parses the text in sentences but lyrics are usually written without punctuation. We added punctuation using \textit{punctuator2} (\cite{tilk2016punctuation}) which also added space for error. 

Lastly, when SPARSAR encountered an unknown word, it failed for the entire song. Since our data were crowd-sourced, it contained many unusual words, so in the beginning it failed on 80\% of our data. We iteratively worked with the authors for months on fixing bugs and adding words (mainly contractions, such as I'mma, y'all, yo', 'em, etc.) into their dictionary. In the end, we were able to successfully run it on 95\% of our data.

However, we were still not very satisfied with the result. As you can see in the example in Table \ref{sparsar_wrong_scheme}, it failed to detect the perfect rhyme between 2\textsuperscript{nd} and 4\textsuperscript{th} line, and instead marked a rhyme on line 1 and 3, where there was none. Such errors were not sparse and led us to believe, that the encoded inclination of this tool to look for sonnet-shaped schemes caused it to make errors in the diverse schemes of song lyrics. It may be a great tool for sonnets, but we have found it insufficient for our purpose, so we proceeded to create our own rhyme detector.


\begin{table}[h!]
	\centering
	\begin{tabular}{c c c} 
		Scheme & Line & \begin{tabular}{@{}c@{}}Last word's pronunciation \\ (as assigned by SPARSAR)\end{tabular}  \\ [0.5ex] 
		\midrule
		a & Pulled out from the station & s-t-ey-sh-ah-n \\ 
		h & fifteen after two & t-uw \\
		a &	300 miles away from Vegas & v-ey-g-ah-s\\
		b & We had nothin better to do & d-uw\\
	\end{tabular}
	\caption{Example of incorrect scheme assignment by SPARSAR. Excerpt from the song \textit{Good Life}.}
	\label{sparsar_wrong_scheme}
\end{table}

\section{Defining the requirements}\label{defining_the_requirements}
Before we dive into algorithm selection and implementation details of our detector, let's define what exactly do we want our detector to do. Additionally, we also establish terms for common cases in Table \ref{terms}, to keep our further explanations short and clear.
\begin{table}[h!]
	\centering
	\begin{tabular}{>{\bfseries}l l} 
		component & a vowel or a consonant cluster in a syllable\\
		rhyme candidates & two lines that are being compared for rhyme \\
		rhyming fellows & two lines that rhyme together\\
		rhyming part & the exact components that participate in rhyme \\&(i.e. are equal or similar in sound)\\
		rhyme group & a group of lines that all rhyme together\\& (i.e. have identical scheme letter) \\
		rhyme rating & rating of the quality of one rhyme between two lines\\
		song rating & rating of the rhyming quality of the entire song\\
	\end{tabular}
	\caption{Establishing terms.}
	\label{terms}
\end{table}

\paragraph{Input}
Although we realize sound can affect rhymes, for the purpose of this thesis we will focus solely on text. As \cite{rhymes_overview} points out, most works focus on rhyme schemes in well structured poetry instead of common rhymes in text. We do not have standard rhyme scheme patterns or fixed syllable counts -- song authors use rhymes arbitrarily. For our input, we expect song lyrics in English, formatted in lines with rhyme always at the end of line. If there will be rhyme in the middle of the line, it will be considered an \textit{\gls{internal_rhyme}} and will not be detected, as we decided in Chapter \ref{chap-related-work} to only focus on \gls{end_rhyme}s in this thesis. Optional empty lines between stanzas or chorus will be preserved but skipped.

\paragraph{Output}
The main element of our output is rhyme scheme. As a single element, it gives the best overview of the song and more importantly, it allows us to compare our detector with others or with the \gls{gold_data}. Additionally, we want more information to assess rhyme quality: rhyme rating for each individual rhyme, song rating, rhyme type, pronunciation of the rhyming part, optional modification of stress, etc.


\section{Pronunciation}
\subsection{Phonetic alphabets overview}
Unlike many other languages, English does not have a straightforward pronunciation rules. Therefore to be able to assess rhymes, we need to transcribe our text into a phonetic alphabet first. There are two commonly used alphabets to choose from -- IPA and ARPAbet. The original International Phonetic Alphabet (IPA) used since 1888 uses one UNICODE character to encode each phoneme and it is commonly used for example in dictionaries. Since it uses non-ASCII characters, ARPAbet was developed as an equivalent for computers. It has two versions: 1-character that uses upper-case and lower-case letters and (the more common) 2-character version where each phoneme is represented by one or more upper-case ASCII characters (\cite{lea1980trends})(see Table \ref{pronunciation_table} for comparison). We will be using the 2-character ARPAbet because it is used by the CMUdict.

\begin{table}[h!]
	\centering
	\begin{tabular}{c c c c} 
		Example word & IPA & 1-character ARPAbet & 2-character ARPAbet \\ [0.5ex] 
		\hline
		st\textbf{o}ry & \textipa{O} & c & AO \\ 
		bu\textbf{tt}er & \textipa{R} & F & DX \\
	\end{tabular}
	\caption{Short comparison of different pronunciation alphabets.}
	\label{pronunciation_table}
\end{table}
\subsection{CMUdict}
Carnegie Mellon University Pronouncing Dictionary (CMUdict) is an open-source pronunciation dictionary.\footnote{\url{http://www.speech.cs.cmu.edu/cgi-bin/cmudict}} Currently it contains 134,373 words (including their inflections) and their pronunciations in 2-character ARPAbet. 
For each word, there is one or several possible pronunciations in North American English including stress markers for primary, secondary or no stress. For the implementation, we used its Python wrapper package \textit{cmudict} \footnote{\url{https://pypi.org/project/cmudict/}}. To use this we had to strip the input of punctuation and convert it to lower case.

CMUdict is a large dictionary and it includes also slang words so it should cover most of our input. To test this, we looked at all last words on each line of our data (since those are the important ones for rhyme analysis) and we found out that 5.52\% of them are not in CMU dictionary. We manually inspected a small portion of them and found out that they can be mostly split into these 5 categories:

\begin{itemize}
	\item uncommon words, e.g. superglue, redundantly
	\item misspelled words, e.g. decsion, girlfren
	\item numbers
	\item foreign words, e.g. revoluccion, ecolli
	\item interjections and onomatopoeia, e.g. shoooshooo, woahwoah
\end{itemize}

\subsection{Dealing with out-of-dictionary words}
In an attempt to decrease the amount of out-of-dictionary words, we replaced the closing quotation mark ``\textipa{’}'' (U+2019) with the typewriter apostrophe ``\texttt{\fontencoding{OT1}\selectfont\symbol{13}}''
 (U+0027) since only the second variant of apostrophe is accepted by CMUdict. However, this only decreased the percentage of unrecognized words to 5.47\%.

Clearly, we needed a way to estimate the pronunciation for these words, so we used grapheme-to-phoneme library \textit{g2p} by \cite{g2pE2019}. It predicts the pronunciation for out-of-dictionary words using deep learning seq2seq model by TensorFlow (\cite{tensorflow2015-whitepaper}).

Even having the pronunciation for every word will not ensure we find every rhyme intended. Song artists may take their liberty in modifying or skewing the pronunciation to make the rhyme work. Sometimes they can also sing two syllables in one beat or use an unusual pronunciation from different culture to convey a message. As we established in the beginning, we will focus only on information retrievable from the text and ignore these possible deviations in pronunciation.

\section{Syllabification}
When comparing lines for rhymes, we have to establish a system of alignment so that we analyze only relevant pairs of phonemes. Initially, we created a simple rhyme detector that just traversed both verses backwards phoneme by phoneme\footnote{For simplification, we use the term phoneme for one symbol in ARPAbet.} and compared them. However, rhyming words do not have to have an equal number of phonemes. For example words in the Table \ref{phon_misalign_table} have a 2-syllable rhyme. If we compared each phonemes one by one they get misaligned on consonant clusters S-T-R and P-L and we will miss the second syllable rhyme. For forced rhymes this inherently becomes a very frequent issue.

\begin{table}[h!]
		\centering
	\begin{tabular}{c r} 
		Word & ARPAbet transcription \\ [0.5ex] 
		\hline
		constrain & K AH N - S \space\space T R EY N \\ 
		complain & K AH\space  M -  P L EY N \\
	\end{tabular}
	\caption{Example of misalignment when aligning by phonemes.}
	\label{phon_misalign_table}
\end{table}

We need to make sure that we are comparing corresponding parts of verses otherwise we will miss the rhyme. A better approach would be to compare corresponding syllables. Each syllable can be further split into 3 groups (``CVC'') -- leading consonant cluster (\textit{onset}), vowel (\textit{nucleus}), and trailing consonant cluster (\textit{coda}). Consonant clusters can sometimes be empty. For syllabification we used Python library \textit{syllabify},\footnote{\url{https://github.com/kylebgorman/syllabify}} which conveniently returns syllables in CVC triplets as described above.



\section{Rhyme analysis}


\subsection{Extracting relevant components}
Rhymes are located at the end of each line so there is no need to analyze the entire verse. How far should we look? The first choice would be to look at the last word. However rhymes can extend over more words as we see in Figure \ref{two-word_rhyme}.
\begin{figure}[htb]\centering
	I was the man the Duke \textbf{spoke to};\\
	I helped the Duchess to cast off his \textbf{yoke, too};\\
	\caption{An example of two-word rhyme from \textit{The Flight of the Duchess} by Robert Browning.} 
	\label{two-word_rhyme}
\end{figure} 

When we look at our rhyme types, they do not go further then the first stressed syllable (looking at the line backwards). Notably, even if the rhyme does extend further we can ignore the rest because it cannot increase the rating. For example, if there are more rhyming syllables preceding the perfect rhyme, they cannot make the score better. Similarly, if the rhyme is not perfect, syllables preceding the final stress would already be considered an \gls{internal_rhyme} -- which is also used (mainly in rap lyrics) but less valued than the classical end rhyme and as stated in section \ref{defining_the_requirements}, not a target of analysis in this thesis. We will therefore limit our window to the minimal number of syllables needed to include the stressed syllable in both rhyme fellows. If one rhyme fellow needs less syllable than other, we will adjust the stress to the right to match the shorter syllable span (and later compensate for this in the rating).


\subsection{Finding similar phonemes}
To set ourselves apart from simple rhyme detectors, we need to detect more than just perfect and imperfect rhymes -- we need to find forced rhymes as well. To determine forced rhyme we need to assess the match in sound of individual phonemes.

For this, we took inspiration from \cite{plechavc2018collocation}. He used the fact that rhymes tend to reoccur and, having enough data, commonly co-occurring pairs are most probably rhymes even without the knowledge of their pronunciation. He calculated the probabilities of phoneme co-occurrences in line-end words from their frequencies in data, used this matrix to predict rhymes, and then iteratively recalculated the matrix using EM algorithm. 

However, our system of alignment is different so our matrix components will look differently. The differences are shown in Table \ref{alignment}. The first difference is that Plechac only has vowels and consonant clusters without acknowledging the syllable separation. We do separate the consonant cluster into two on the border of syllable. The second difference is that Plechac recognized the position of the component and pairs only the components on the same position (shown as color). We do not believe that position has an effect on phoneme similarity and therefore we add it together for all positions, e.g. is once two phonemes co-occur on 3\textsuperscript{rd} position and once on the 5\textsuperscript{th} we will count that this pair of phonemes has 2 co-occurrences.

\begin{table}[h!]
	\centering
	\begin{tabular}{c r} 
		Plechac & \textipa{\color{blue}k \color{magenta}A: \color{PineGreen} r.p \space\space \color{BurntOrange} @ \color{BrickRed}  n.t \space\space\color{Cerulean} @ \color{Fuchsia}r}\\
		Ours & \textipa{\color{blue}k \color{magenta} A: \color{blue}r \space .p \color{magenta} @ \color{blue} n \space .t \color{magenta} @ \color{blue}r} \\
	\end{tabular}
	\caption[Comparison of alignments]{Comparison of our components for word \textit{carpenter} with those of \cite{plechavc2018collocation}. Different color signifies different component group.} 
	\label{alignment}
\end{table}

Another difference is that Plechac only looks at the last word and 1 to 2 syllables while we look at everything after the last stress up to 4 syllables.

Knowing that perfect match in sound always means perfect rhyme, we froze the values on matrix diagonal to reflect it. In \cite{plechavc2018collocation}, the diagonal was being recalculated as the rest of the matrix, and although it converged to high numbers we felt it did not reflect the superiority of the perfect match.

To calculate the probabilities in the matrix, we used the formula from \cite{plechavc2018collocation} adding the adjustments described above. The formula is following: 


	    \[ p(i,j) = \begin{cases} 
	    \mbox{1.0,} & \mbox{if}  i=j \\ 
	    \mbox{...,} 
	     & \mbox{otherwise} \\
	    \end{cases} \]

where\textit{ f(i,j)} represents the frequency of the co-occurrence of the pair of components \textit{i} and \textit{j}, and similarly \textit{f(i)} represents the total number of occurrences of the component \textit{i}. \textit{p(i,j)} is then the value in the matrix on coordinates i, j.


\subsection{Rhyme rating}
 We will first calculate rhyme ratings for pairs of verses and then use them to calculate an overall song rating. Not only do these rhyme ratings help us evaluate the rhyming quality of the song but they might also be an interesting feedback for the writer. 
 
 For each rhyme, we would like to assign a rating between 0 and 1. Since perfect rhyme is often in literature described as superior because it represents the perfect match in sound it will logically receive the highest rating of 1.
 
 For the cases, such as imperfect rhymes or some forced rhymes, where it was necessary to move the stress to the rhyming part, we will give a penalty of -0.1 for this imperfection. 
 
	When the phoneme sounds are similar, we will assign rhyme rating as a simple multiplication of probabilities for the individual components from the matrix.
	
	\[rhyme\ rating = \prod_{i,j\ \epsilon\ rhyming\ part} p(i,j) \]


\section{Training the detector}
Since we are using the EM algorithm to train our detector for recognizing similar phonemes, the outline of the algorithm is as follows:
\begin{enumerate}
	\item Initialize the matrix of collocation probabilities.
	\item Find rhymes using this matrix.
	\item Adjust the matrix based on detected rhymes.
	\item Repeat steps 2 and 3 until convergence.
\end{enumerate} 

For the matrix initialization, Plechac calculates the collocations in the data and preserves only pairs with number of collocations above a set threshold. We instead initialized it with default value of 0.2. 

\paragraph{}

\section{Scheme}
\subsection{Finding all rhymes}
To search for rhymes in the full lyrics we need to decide which verse pairs to check. The most straight-forward approach would be ``brute force'' -- try each line with all the other lines. Besides its obvious disadvantage of increased time requirements it also detects rhymes that span across tens of lines. It is not strictly defined how many lines apart can the rhyme fellows be to still be considered a rhyme -- the author can even make it a part of his artistic expression  e.g. in ``Author's Prologue'' by \cite{thomas1952author} the 1\textsuperscript{st} line rhymes with 102\textsuperscript{th}, 2\textsuperscript{nd} with 101\textsuperscript{th} and so on. Realistically, a rhyme between a line at the beginning of the lyrics and 20 lines later would not have a strong effect on the song listener -- it requires a close proximity of rhyme fellows within the poem for the rhyme to be noticed by ear. Since the most common stanzaic form in English is a quatrain, a stanza of four lines (\cite{eastman1970norton}), we decided to set the default window size to 5. 

We traversed the song lyrics from beginning to the end, line by line, for each line trying all rhyme candidates in the given window.

Some words may have multiple possible pronunciations -- in that case we evaluate each possible combination of pronunciations for all words in the given line pair. For every such combination we assign rhyme rating and keep only those with rhyme rating above the selected threshold.
\todo[inline]{\cite{rhymes_overview}- where to draw the line between intended rhyme and accidental word similarity} For default, we have set this threshold to 0.8 because we found it to work quite well in our data. However, it can be adjusted by user to their value of choice, similarly with other parameters such as window size.  

From this list of candidates for each line, we select only the candidate with the highest rhyme rating. When the ratings are identical, we select the closest line. Other candidates are saved, in case changes need to be made later in scheme adjustment phase. When the line doesn't have any suitable rhyme candidates, it is assigned rating 0. If the line rhyme with a candidate that is already a part of a rhyme, they join together into a rhyme group of 3 (or more) lines.
 



\subsection{Assigning rhyme scheme}
 Rhymes in songs or poems are typically marked using a rhyme scheme. That means each verse gets assigned a letter -- lines that share the same letter rhyme and those with different letters do not. We also decided to adapt this common notation. In case the song needs more letters than there are in the alphabet we will add another letter and continue alphabetically -- a, b, c, ..., aa, ab, ac, ..., ba, bb, bc, ..., ca, etc.
 
 There are two options for representing non-rhyming lines. The first is to assign every non-rhyming line the same default character. We chose this option as the default, because it is easier to read for the user. The second option is to assign each non-rhyming line a unique rhyme scheme letter. This approach is more suitable for metrics that look at rhyme scheme letters as clusters of rhyming words. We support both options and can convert one the other when needed.

\subsection{Scheme adjustment}
In some cases, the algorithm of selecting the best rhyme for each line does not yield the best possible scheme. Consider, for illustration, the example in Table \ref{scheme_adjustment}. There is a perfect rhyme between lines 1-2 and 3-4, and a forced rhyme between lines 2-3. With the algorithm as is, it would receive \textit{aaaa} scheme. However, that is not what a human, for example the gold data annotator, would assign. He would see that similarity inside the first and the second tuple is greater so they logically form two rhyme groups and the scheme is \textit{aabb}. 

Additionally, song rating for scheme \textit{aaaa} would be less than 1 (because of the forced rhyme rating) while the song rating for \textit{aabb} would be equal to 1. Loosing rating by marking weaker rhymes does not make sense so we must add an exception to only keep the better score. We can see that this problem is similar to the problem of maximizing song rating. 

\begin{table}
	\begin{tabular}{l l}
		1&Packs of Backwoods and Dutches, leave the Swishers for the \textbf{sweeties}  \\
		2&Only roaches in the dishes we be ripping up your \textbf{beedies}  \\
		3&We be ripping up your treaties, I ain't ripping if it's \textbf{seedy}  \\
		4&I ain't riffing, I ain't raffing, I'm just rapping on a \textbf{CD} \\
	\end{tabular}
	\caption[Scheme adjustment example.]{Example for scheme adjustment from \textit{aaaa} to \textit{aabb}. Excerpt from the song \textit{Whooping Cough} from  our dataset.}
	\label{scheme_adjustment}
\end{table}

To address this issue, we have to do one more iteration over the resulting rhyme groups. We focused only on rhyme groups of size 4 and larger because for smaller group such changes would not increase the score. For a larger group, we iterate over the rhymes, starting from the ones with the lowest rating, and tried how would removing this rhyme affect the song rating. If it increased the song rating, we kept the change and split the rhyme group as necessary. If the rating did not increase, we tried removing the next rhyme, and if we ran out of rhymes to remove we kept the group as is.

\section{Calculating song rating}
The next step is to combine these rhyme ratings into one final rating for the entire song. We will use the straight-forward approach of averaging the assigned rhyme ratings. The dilemma here is where to store the rhyme rating since rhyme is a property of two lines. The first logical idea would be to store it with each line participating in the rhyme. However, then we would add it to the final song rating twice. Moreover, for larger rhyme groups, it would be disproportionate, because third and all the following lines would be added only once. 

Therefore we decided to store the rating only with the second line of the rhyme. That means the first line in each rhyme group will always be assigned the default value ``-''. It cannot be assigned rhyme rating 0 because that would mean it is a non-rhyming line and would lower the final average. In summary, song rating is the average of rhyme ratings for all rhymes except the lines with unassigned rating ``-''.






\section{Syllables}
\todo[inline]{check if this wouldn't work with g2p}
Since meter plays an important role in rhymes, another relevant property to examine is the number of syllables. To count syllables for each line we used Python package \textit{syllabify} \footnote{\url{https://github.com/kylebgorman/syllabify}} which returns syllables using ARPAbet transcription. For words not in CMUdict we used a simple heuristic -- since the nucleus of each syllable is most often a vowel (except for syllabic consonants) we counted the number of (groups of) vowels and used it as an estimation for the number of syllables. Although this gives a wrong estimate for some words e.g. \textit{rhythm} or \textit{house}, it performed quite well when we tested it on a few out-of-dictionary words from our dataset. We found it unnecessary to try to further improve the heuristic because the words that are not in CMUdict are often foreign words that do not follow the standard pronunciation rules of English so any application of these rules would probably be of little help.


\chapter{Evaluation}\label{evaluation}
In this chapter, we will evaluate our rhyme detector. In the beginning, we will compare its performance with RhymeTagger. Then we will use it to analyze our dataset and calculate statistics about song lyrics.

\section{Performance evaluation on schemes}
To evaluate the quality of our Rhyme Detector, we will compare its rhyme scheme predictions with \gls{gold_data}. We have two annotated datasets we can use -- Chicago Rhyming Poetry Corpus and  the annotated subset of our own dataset.
\MP{Takže dev set, nebo test set?}
We will also compare our performance with that of RhymeTagger.

\subsection{Taggers}
We will be comparing different variants of the taggers to also evaluate the influence some factors have on the performance. The variants are following:
\begin{itemize}
	\item \textbf{RhymeTagger} -- the original version of RhymeTagger, as it was trained by Plecháč.
	\item \textbf{RhymeTagger -- fine-tuned} -- RhymeTagger trained on one third of our data.
	\MP{Proč jen třetina? Předpokládám, že trénování RhymeTaggeru bylo pomalé (pomalejší než Rhyme Detectoru?) či vyžadovalo víc paměti, než jste měla k dispozici.
	Je-li to tak, napište to sem, třeba do poznámky pod čarou.
	Metodologicky ideální by bylo ještě natrénovat i váš Detector jen na té třetině, aby to bylo porovnatelné.}
	\MP{To jste trénovala zcela znovu, že? Tedy se nejedná o fine-tuning, ale training from scratch.
	Tedy by šlo RhymeTagger přejmenovat na RhymeTagger(ChicagoRPC) a fine-tuned na RhymeTagger(Genius$\frac{1}{3}$), kde ChicagoRPC a Genius by se zadefinovalo jako názvy těch datasetů,
	respektive jejich trénovacích částí.}
	\item \textbf{Rhyme Detector} -- our detector, as described in Chapter \ref{chap-rhyme-analysis}, with \added{the} default settings
	\MP{Kolik iterací EM to tedy bylo? A dosáhlo to opravdu konvergence?}
	\item \textbf{Rhyme Detector -- experiment} -- an experimental version of our detector\deleted{,} that was not trained with \added{the} EM algorithm, but instead the matrix was created by counting the co-occurrences in the data and using their probabilities
	\MP{Pokud si pamatuji, tak zde jste ale co-occurrences nepočítala z rhyme fellows,
	ale ze všech adeptů (tedy rhyme candidates) v rámci daného window.
	}
	\item \textbf{Rhyme Detector -- 1 iteration} -- our detector after \added{the} 1\textsuperscript{st} iteration of \replaced{the EM algorithm}{training}
	\item \textbf{Rhyme Detector -- perfect} -- our detector \replaced{configured to}{using the setting of} only finding perfect rhymes -- the co-occurrence matrix is an identity matrix
\end{itemize}

\subsection{Scores}
To evaluate the performance, we need to find an appropriate measure that could compare the gold scheme with the prediction. This task may seem easy at first, but we need to be careful because the straight-forward approach of comparing the schemes letter by letter would not work. If the prediction made an error in the beginning it would alphabetically shift the rest of the scheme and it would no longer match with the gold data.

\paragraph{Last Index Score (LI)} To solve this problem, we propose Last Index Score (LI score). The idea is to convert the scheme from letter representation to last rhyme's index representation. This means for each line, we will use the index of the last line that rhymed with it. If the line does not have a rhyme, we will use index -1. With such representation, we can compare these \replaced{indices}{indexes} directly, independently from scheme letters, and the proportion of matching \replaced{indices}{indexes} will give us a score between 0 and 1. For an illustrative example, see Table \ref{LI_score_example}.
\MP{Dovolil jsem si přidat do tabulky sloupec line index a trochu přeformátovat}

% \begin{table}[h!]
% 	\centering
% 	\begin{tabular}{c c | c c}
% 		\multicolumn{2}{c |}{Straight-forward} &
% 		\multicolumn{2}{c }{Last Index}\\
% 		\hline
% 		Gold & Prediction & Gold & Prediction\\
% 		\hline
% 		\color{OliveGreen}a		&\color{OliveGreen}a		&\color{OliveGreen}-1		&\color{OliveGreen}-1	\\
% 		\color{Bittersweet}a		&\color{Bittersweet}b		&\color{Bittersweet}0		&\color{Bittersweet}-1	\\
% 		\color{Bittersweet}b		&\color{Bittersweet}c		&\color{OliveGreen}-1		&\color{OliveGreen}-1\\
% 		\color{Bittersweet}b		&\color{Bittersweet}c		&\color{OliveGreen}2		&\color{OliveGreen}2	\\
% 		\color{Bittersweet}c		&\color{Bittersweet}d		&\color{OliveGreen}-1		&\color{OliveGreen}-1	\\
% 		\color{Bittersweet}c		&\color{Bittersweet}d		&\color{OliveGreen}4		&\color{OliveGreen}4	\\
% 		\color{Bittersweet}b		&\color{Bittersweet}c		&\color{OliveGreen}3		&\color{OliveGreen}3	\\
% 		\color{Bittersweet}b		&\color{Bittersweet}c		&\color{OliveGreen}6		&\color{OliveGreen}6	\\
% 		\midrule
% 		\multicolumn{2}{c |}{SCORE: 0.125} &
% 		\multicolumn{2}{c }{SCORE: 0.875}\\
% 	\end{tabular}
% 	\caption[Comparison of the straight-forward approach and LI score.]{Comparison of the straight-forward approach and LI score.}
% 	\label{LI_score_example}
% \end{table}
\begin{table}[h!]
	\centering
	\begin{tabular}{c cc cc}\toprule
	 line &
		\multicolumn{2}{c}{Straight-forward} &
		\multicolumn{2}{c}{Last Index}\\\cmidrule(r){2-3}\cmidrule(l){4-5}
      index    & Gold & Prediction & Gold & Prediction\\\midrule
		0 & \color{OliveGreen}a		&\color{OliveGreen}a		&\color{OliveGreen}-1		&\color{OliveGreen}-1	\\
		1 & \color{Bittersweet}a		&\color{Bittersweet}b		&\color{Bittersweet}0		&\color{Bittersweet}-1	\\
		2 & \color{Bittersweet}b		&\color{Bittersweet}c		&\color{OliveGreen}-1		&\color{OliveGreen}-1\\
		3 & \color{Bittersweet}b		&\color{Bittersweet}c		&\color{OliveGreen}2		&\color{OliveGreen}2	\\
		4 & \color{Bittersweet}c		&\color{Bittersweet}d		&\color{OliveGreen}-1		&\color{OliveGreen}-1	\\
		5 & \color{Bittersweet}c		&\color{Bittersweet}d		&\color{OliveGreen}4		&\color{OliveGreen}4	\\
		6 & \color{Bittersweet}b		&\color{Bittersweet}c		&\color{OliveGreen}3		&\color{OliveGreen}3	\\
		7 & \color{Bittersweet}b		&\color{Bittersweet}c		&\color{OliveGreen}6		&\color{OliveGreen}6	\\
		\midrule
		& \multicolumn{2}{c}{SCORE: 0.125} &
		\multicolumn{2}{c}{SCORE: 0.875}\\\bottomrule
	\end{tabular}
	\caption[Comparison of the straight-forward approach and LI score.]{Comparison of the straight-forward approach and LI score.}
	\label{LI_score_example}
\end{table}


\paragraph{ARI score} If we look at the rhyme scheme, we can notice that scheme letters can equally be represented as line cluster. All lines that share a scheme letter form one cluster (a rhyme group), and every line that does not have a rhyme is a cluster of its own. Adjusted Rand Index (ARI) Score\footnote{\url{https://en.wikipedia.org/wiki/Rand\_index\#Adjusted\_Rand\_index}} is a corrected-for-chance statistical measure of similarity between two data clusterings. We can therefore convert the schemes to use a unique letter for each non-rhyming line and use this score to evaluate the similarity of the schemes. 

\paragraph{} For these two scores, we include both micro and macro average on the dataset. Macro average is the average of all song scores, while micro average is the average of song scores weighted by the number of lines in each song.


\subsection{Comparison of the results}
We calculated the aforementioned scores for all variants of tagger\added{s} and summed up the results in Tables \ref{reddy_eval} and \ref{my_eval}. 

They both performed better on the Chicago Poetry Corpus (see Table \ref{reddy_eval}). Surprisingly, our detector performed the best after 1\textsuperscript{st} iteration, but the difference is not \deleted{very} significant.
\MP{Signifikance je přesně matematicky definovaná, měl by se uvést i test, kterým se počítala a alpha.
Nelze psát very/more significant.
Pokud nebyla exaktně měřena signifikance, lze o tom rozdílu psát třeba substantial nebo large.}

\begin{table}[h!]
	\centering
	\begin{tabular}{l | r r | r r}
	&	\multicolumn{2}{c |}{ARI} &
		\multicolumn{2}{c }{LI}\\
	\cline{2-5}	
	&macro &  micro  & macro  & micro \\
	\midrule
	RhymeTagger & \textbf{0.9463} & \textbf{0.9384} & \textbf{0.9635} & \textbf{0.9534} \\
	RhymeTagger fine-tuned& 0.8819 & 0.8822 & 0.9282 & 0.9202 \\
	\midrule
	Rhyme Detector & 0.9040 & 0.8915 & 0.9413 & 0.9272 \\
	Rhyme Detector -- experiment & 0.9025 & 0.8906 & 0.9406 & 0.9272 \\
	Rhyme Detector -- 1 iteration & \textbf{0.9066} &\textbf{0.8926} &\textbf{0.9426} &\textbf{0.9277} \\
	Rhyme Detector -- perfect &0.8824 &0.8732 &0.9293 &0.9149 \\
 \end{tabular}
	\caption{Evaluation of taggers on Chicago Rhyming Poetry Corpus.}
	\label{reddy_eval}
\end{table}

On our dataset (Table \ref{my_eval}), however, the best performing was the experiment variant.
\MP{No, celkově nejlepší byl ve všech 4 evaluacích původní RhymeTagger.}
Surprisingly, the fine-tuned RhymeTagger performed worse than the original pre-trained version. The possible cause is the nature of our data -- song lyrics have a significantly smaller percentage of rhymes than poems, so the collocations approach might not work as well for our dataset. This might also be the reason why our detector, trained only on our data, does not perform as well as RhymeTagger.
\MP{Nabízí se otázka, proč jste tedy nenatrénovala Váš detector na ChicagoRPC.
Můžete té otázce předejít, že to poznamenáte jako future work.
Případně to můžete ignorovat a čekat, zda se Vás na to nezeptá oponent
a nestihnete to ještě do obhajoby.}
\MP{Těch otázek k diskutování se nabízí více.
Např. nemáte-li v tom EM chybu, když to nikdy není nejlepší z Vašich verzí.
A obecně: čím jsou způsobené ty rozdíly.
Teď už na tyto otázky odpovědi nalézt nestihneme.
Je ale možné, že se na něco takového zeptám ve svém posudku.}

\begin{table}[h!]
	\centering
	\begin{tabular}{l | r r | r r}
		&	\multicolumn{2}{c |}{ARI} &
		\multicolumn{2}{c }{LI}\\
		\cline{2-5}	
		&macro &  micro  & macro  & micro \\
		\midrule
	RhymeTagger &  \textbf{0.6519} &\textbf{0.6627} &\textbf{0.8021} & \textbf{0.8139} \\
	RhymeTagger fine-tuned& 0.6309 &0.6443 &0.7883 &0.8040 \\
	\midrule
	Rhyme Detector & 0.6157 &0.6306 &0.7716 &0.7861 \\
	Rhyme detector -- experiment &\textbf{0.6359} &\textbf{0.6571} &\textbf{0.7866} &0.8020 \\
	Rhyme detector -- 1 iteration& 0.6096 &0.6256 &0.7682 &0.7832 \\
	Rhyme Detector -- perfect &0.6224 &0.6410 &0.7838 &\textbf{0.8030}\\
	\end{tabular}
	\caption{Evaluation of taggers on our annotated dataset.}
	\label{my_eval}
\end{table}



\section{Statistical analysis of the dataset}
Although our detector was trained on our dataset, it was unsupervised so we can still use our detector to evaluate this dataset and give us new statistical information about a large number of song lyrics. We ran rhyme detection for nearly half a million songs and summed up the results in Tables \ref{rhyme_line_stats}, \ref{rhyme_types_perc}, \ref{rhyme_group_size}, \ref{song_rating_stats}, and \ref{rhyme_stats}. In the rest of this section, we will look at them more closely and discuss the outcome that might be surprising, or the opposite, confirms the specific  characteristics of a particular genre. Extreme values are emphasized in the tables. Keep in mind, that the lyrics and their classification to genres is crowd-sourced and might be biased.
% \begin{table}[h!]
% 	\centering
% 	\begin{tabular}{l | r r r r r} 	
% 		Genre & 			Pop & 		Rap & 		Rock & 		R\&B & 		Country\\ 
% 		\midrule
% 		 Total songs& 293,679 & 99,185& 34,372& 5,125& 3,816 \\
% 		Total lines& 9,104,273 &5,661,603& 1,087,245& 225,344& 121,207 \\ 
% 		Total rhyming lines& 4,536,554& 2,849,905& 523,879& 117,862& 61,142 \\ 
% 		 Rhyming lines (\%) & 49.8\%& 50.3\%& 48.2\%& 52.3\%& 50.4\%  \\
% 		 Average lines per song & 31.001 & \textbf{57.081} & 31.632 & 43.970 & 31.763  \\
% 
% 	\end{tabular}
% 	\caption{General statistics about dataset and rhymes, per genre.} 
% 	\label{rhyme_line_stats}
% \end{table}
\MP{Opět jsem si dovolil předělat tabulku, tentokrát včetně transpozice,
aby ve sloupcích pod sebou byla čísla stejného typu, což usnadňuje porovnávání a je to základní pravidlo pro tabulky.
Bohužel jsem tím narušil jednotu -- v některých tabulkách jsou teď žánry ve sloupcích a v jiných v řádcích, ale přišlo mi to jako menší zlo.
Tedy šlo by i ty zbývající transponovat, aby byly všude žánry v řádcích, což by asi bylo nejlepší, ale na to jsem neměl čas.
}
\begin{table}[h!]\centering
\begin{tabular}{l  r r r r r}\toprule
       &         & \multicolumn{4}{c}{Lines} \\\cmidrule(l){3-6}
\pulrad{Genre}& \pulrad{Songs}
                 & total     & avg per song  & rhyming   & (\%) \\\midrule
Pop    & 293,679 & 9,104,273 & 31.001        & 4,536,554 & 49.8 \\
Rap    &  99,185 & 5,661,603 &\textbf{57.081}& 2,849,905 & 50.3 \\
Rock   &  34,372 & 1,087,245 & 31.632        &   523,879 & 48.2 \\
R\&B   &   5,125 &   225,344 & 43.970        &   117,862 & \textbf{52.3} \\
Country&   3,816 &   121,207 & 31.763        &    61,142 & 50.4 \\\bottomrule
\end{tabular}
\caption{General statistics about dataset and rhymes, per genre.} 
\label{rhyme_line_stats}
\end{table}

In Table \ref{rhyme_line_stats}, we sum up the basic information for all genres including the portion of lines that rhyme and we can already see some interesting results. Surprisingly, the highest portion of rhyming lines is in the R\&B genre. We do not see any characteristic of this genre that could cause this. However, it is not a big difference and maybe having more examples from this genre would make it less significant. 

We can see that throughout genres typically only about half of the lines rhyme. This shows\deleted{,} that rhyming in songs is not as essential as perhaps in poems.
\MP{Taky by to mohlo být kvůli chybám detektoru.
Kdyby v tabulce \ref{my_eval} bylo i precision a recall a recall bylo vysoké,
tedy většina rýmů je detekována, tak by tím šlo námitku o chybách detektoru částečně vyvrátit.
Úplně nejlepší by ale bylo změřit procento rhyming lines ve zlatých datech ve Vašich písních
a v Chicago básních a ověřit, zda i tam platí Vaše hypotéza, že básně se rýmují víc než písně.
}
\MP{Další možný důvod nižšího množství detekovaných rýmů v písních může být ten,
že Vaše data nejsou ``správně'' formátována, tedy nemají vždy rým na konci řádky, jak předpokládáte.
Třeba ve Figure 5.2 v písni Love Game se zřejmě rýmují řádky 4 a 5 s kiss you a miss you,
ale za tím druhým je ještě ``, babe'', takže ten rým Vašemu detektoru unikne.
Tohle byste zde (v textu či poznámce pod čarou) případně taky mohla zmínit.
Tipuji, že docela dost z těch řádků bez detekovaných rýmů může být způsobeno právě tímto.
}
Predictably, rap has a significantly higher average number of lines per song\added{,} which confirms the fact that this genre is more talkative. What may be unexpected is that it is nearly two times more than for the other genres -- only R\&B slightly stands out but that is not a surprise because it has been influenced by rap.
\MP{Moc se v tom nevyznám, ale dle Wikipedie "Rap was used to describe talking on records as early as 1971" a "Rhythm and blues ... is a genre of popular music that originated in African-American communities in the 1940s."
Tak bych řekl, že spíš R\&B ovlivnilo rap, pokud už se nebere rap jako součást hip hopu a ten jako součást moderního R\&B.
Možná by šlo napsat, že ty styly jsou podobné, a nebo to "but that is not a surprise..." vypustit.}
\MP{Při diskuzi délky by šlo analyzovat, jak moc které žánry obsahují nějaké opakující se sloky/refrény.
Zásadní pak je, zda jsou tato opakování vypsána ve vašich textech.
Toto druhé by šlo detekovat automaticky, to první možná taky, pokud se tam píše aspoň "Ref".
Analýzu už nestihnete, ale víte-li, že třeba v country jsou refrény časté, ale v datech typicky/nikdy nejsou vypasné, můžete to napsat do poznámky pod čarou.}
% \begin{table}[h!]
% 	\centering
% 	\begin{tabular}{l | r r r r r} 	
% 		Genre & 			Pop & 		Rap & 		Rock & 		R\&B & 		Country\\ 
% 		\midrule
% 		Perfect masculine &	72.5& 	\textbf{58.2}& 	72.3& 	70.2& 	73.5 \\
% 		Perfect feminine &	7.9&		8.4& 		7.7& 		8.5& 		6.2 \\
% 		Perfect dactylic & 	0.7 &		0.5 & 	0.9 &		0.5& 		0.3 \\  
% 		Imperfect & 		12.0& 	\textbf{22.3} & 	12.1 & 	13.5 & 	12.2 \\
% 		Forced &  			6.9 & 	\textbf{10.6} & 	7.0 & 	7.3 &		7.8 \\
% 	\end{tabular}
% 	\caption{Percentage of different rhyme types from all rhymes in the dataset, per genre.} 
% 	\label{rhyme_types_perc}
% \end{table}

\begin{table}[h!]\centering
\begin{tabular}{l r r r r r}\toprule
                     & \multicolumn{5}{c}{Genre} \\\cmidrule{2-6}
 \pulrad{Rhyme type} & Pop  & Rap     & Rock & R\&B & Country\\\midrule
 Perfect             & 81.1 &\bf 67.1 & 80.9 & 79.2 & 80.0 \\
 \quad --- masculine & 72.5 &    58.2 & 72.3 & 70.2 & 73.5 \\
 \quad --- feminine  &  7.9 &     8.4 &  7.7 &  8.5 &  6.2 \\
 \quad --- dactylic  &  0.7 &     0.5 &  0.9 &  0.5 &  0.3 \\  
 Imperfect           & 12.0 &\bf 22.3 & 12.1 & 13.5 & 12.2 \\
 Forced              &  6.9 &\bf 10.6 &  7.0 &  7.3 &  7.8 \\\bottomrule
\end{tabular}
\caption{Percentage of different rhyme types from all rhymes in the dataset, per genre.} 
\label{rhyme_types_perc}
\end{table}
\MP{Opět jsem upravil tabulku a přidal součet pro všechny typy perfect.}

Next, Table \ref{rhyme_types_perc} shows distribution of different rhyme types. It did not come as a surprise that the most common type, by a long shot, is perfect masculine. The reasons behind this might be several -- not only has perfect match the strongest effect melodically, it is also the easiest to come up with, and makes the lyrics easy to remember. The multi-syllable perfect rhymes have a lower percentage as longer matching word pairs are rather rare. The amount of forced rhymes might be higher in reality because their detection is the hardest and they might be missed more often.
\MP{To by šlo změřit na testsetu (byť dost nepřesně, protože je na to malý, ale ten Chicago by šel)
jaká je úspěšnost (a precision a recall) na kterém typu rýmů
a zda jsou opravdu forced nejtěžší.
Opět: tu analýzu teď nestihnete, tak to nechte; možná se na to zeptám v posudku.}

Concerning rhyme types, we see that genres are generally not very different, except for rap. Rap is very unique with rhymes, its artists are known for playing with them more creatively, using \gls{internal_rhyme}s, consonance, and assonance more often. They frequently play with emphasis what can be seen as a rapid increase in imperfect rhymes. There are more forced rhymes as well and perfect rhymes are decreased as a result.

% \begin{table}[h!]
% 	\centering
% 	\begin{tabular}{l | r r r r r} 	
% 		Genre & 			Pop & 		Rap & 		Rock & 		R\&B & 		Country\\ 
% 		\midrule
% 		2-syllable rhymes & 91.1& 90.3& 91.1& 90.6 & 93.3\\
% 		5-syllable rhymes& 8.2& 9.2& 8.0& 8.9& 6.4  \\
% 		8-syllable rhymes& 0.7& 0.5& 0.9& 0.5& 0.3 \\
% 		Perfect sound match & 93.1&\textbf{ 89.4}& 93.0& 92.7& 92.2  \\
% 		Stress moved & 14.5& \textbf{28.3}& 14.5& 16.5& 14.8 \\
% 		
% 	\end{tabular}
% 	\caption{Statistics about rhyme properties in general, disregarding rhyme types, in percentage from total rhymes.} 
% 	\label{rhyme_stats}
% \end{table}
\begin{table}[h!]\centering
\begin{tabular}{l r r r r r}\toprule
                                 & \multicolumn{5}{c}{Genre} \\\cmidrule{2-6}
 \pulrad{Rhyme category}         & Pop  & Rap     & Rock & R\&B & Country\\\midrule
 1-syllable (2-component) rhymes & 91.1 &    90.3 & 91.1 & 90.6 & 93.3 \\
 2-syllable (5-component) rhymes &  8.2 &     9.2 &  8.0 &  8.9 &  6.4 \\
 3-syllable (8-component) rhymes &  0.7 &     0.5 &  0.9 &  0.5 &  0.3 \\
 Perfect sound match             & 93.1 &\bf 89.4 & 93.0 & 92.7 & 92.2 \\
 Stress moved                    & 14.5 &\bf 28.3 & 14.5 & 16.5 & 14.8 \\\bottomrule
\end{tabular}
\caption{Statistics about rhyme properties in general, disregarding rhyme types, in percentage from total rhymes.} 
\label{rhyme_stats}
\end{table}
\MP{Opět jsem upravil tabulku a opravil nesmyslné 8-syllable.}

Table \ref{rhyme_stats} is quite similar to the previous table, but by counting syllables regardless of rhyme type, and evaluating sound match and stress separately, it offers us a little bit different angle. By seeing that the percentages of \replaced{3-syllable}{8-syllable} rhymes match the percentages we have seen in Table \ref{rhyme_types_perc}, we can assume that \replaced{3-syllable}{8-syllable} rhymes might be exclusively perfect.
\MP{Kdyby to nebyl pefect rhyme, tak přesunete přízvuk doprava a už nebude započítán jako tříslabičný, ne?
Leda by tam byl nějaký forced rhyme, kde je podobnost zvuků na třetí slabice velmi vysoká,
 takže se ho vyplatí zahrnout radši než utrpět 0.1 stress penalty,
 pokud to tedy takto opravdu porovnáváte.
Každopádně ale ve skutečnosti by mohlo být těch 3-slabičných více,
 jen je jako 3 slabičné nepočítáte.
Nebo snad ty kategorie ``3-syllable (8-component)'' počítáte podle původních přízvuků ještě před případným přesunem přízvuku?
Pak ale není jasné, co děláte s rýmy, kde má každý fellow jiný počet slabik po přízvuku.
}
\MP{Zkusil jsem do Vašeho detektoru zadat ``něco kiss you -- něco miss you'' a zanalyzovalo to jako masculine, jak jsem čekal.
Stejně tak Váš příklad z 3.4.2 ``spoke to -- yoke, too'' se zanalyzuje jako masculine.
Dle Vaší definice je to tak správně -- ve slovníku je přízvuk na začátku každého slova
a neřešíte, že některá slova (obzvlášť krátká a neplnovýznamová) lze vyslovit i bez přízvuku,
respektive dohromady s předchozím slovem.
To může být další důvod, proč detektor nachází tak málo víceslabičných rýmů, byť v datech jsou.
Opět: opravit to nestíháme, šlo by to ale zmínit někde v textu -- a dát najevo, že o tom víte.
}
The decreased match in sound and increased moving of stress in rap confirm the unique properties of rap we have seen previously.

A slightly increased percentage of \replaced{1-syllable rhymes}{2-syllable} in country may be noteworthy but we see no significant properties of country that could support this as a general claim.

% \begin{table}[h!]
% 	\centering
% 	\begin{tabular}{l | r r r r r} 	
% 		Genre & 			Pop & 		Rap & 		Rock & 		R\&B & 		Country\\ 
% 		\midrule
% 		Average groups per song& 6.134 &\textbf{11.484} &6.091 &8.620 &6.676  \\
% 		Average groups per 100 lines &19.787 &20.119 &19.255 &19.605 &\textbf{21.018} \\
% 		Max groups per song & 169 &224 & 81 & 48 &98\\
% 		Average group size & 2.518 &2.502 &2.502 &2.668 &\textbf{2.400} \\
% 		Max group size & 159 &98 & 68 & 42 & 24\\
% 	\end{tabular}
% 	\caption{Statistics about rhyme group size per genre.} 
% 	\label{rhyme_group_size}
% \end{table}

\begin{table}[h!]\centering
\begin{tabular}{l rr r rr}\toprule
       & \multicolumn{2}{c}{groups per song}
                        & groups per & \multicolumn{2}{c}{group size} \\\cmidrule(r){2-3}\cmidrule(l){5-6}
\pulrad{Genre}&avg& max & 100 lines&     avg &   max \\\midrule
Pop    &     6.13 & 169 &    19.79 &    2.52 &   159 \\
Rap    &\bf 11.48 & 224 &    20.12 &    2.50 &    98 \\
Rock   &     6.09 &  81 &    19.26 &    2.50 &    68 \\
R\&B   &     8.62 &  48 &    19.61 &    2.67 &    42 \\
Country&     6.68 &  98 &\bf 21.02 &\bf 2.40 &\bf 24 \\\bottomrule
\end{tabular}
\caption{Statistics about rhyme group size \added{and counts} per genre.} 
\label{rhyme_group_size}
\end{table}


Table \ref{rhyme_group_size} summarizes statistics concerning size of rhyme groups. We can observe nearly double average size for rap compared to other genres, which directly corresponds to nearly double average song length, as we have seen in Table \ref{basic_stats}. 

An interesting observation can be made \replaced{about}{for} country -- average number of rhyme groups per 100 lines is slightly higher than for other genres. This corresponds with average group size being lower -- obviously country tends to contain more and smaller rhymes groups. It would be interesting to know\deleted{,} whether this is only a property of our dataset or a property of country music in general. Although \added{the} maximum group size does not tell us any general information about the group because it may only be an outlier, \deleted{but} it is still interesting to see\deleted{,} that this number is again the smallest for country.

% \begin{table}[h!]
% 	\centering
% 	\begin{tabular}{l | r r r r r} 	
% 		Genre & 			Pop & 		Rap & 		Rock & 		R\&B & 		Country\\ 
% 		\midrule
% 		Average song rating& 0.432 & \textbf{0.599} &0.420 &0.520 &0.456  \\
% 		Median & \textbf{0.521} & 0.380 &0.357 &0.235 & 0.269\\
% 	\end{tabular}
% 	\caption{Song rating per genre.} 
% 	\label{song_rating_stats}
% \end{table}
\begin{table}[h!]\centering
\begin{tabular}{l r r r r r}\toprule
                    & \multicolumn{5}{c}{Genre} \\\cmidrule{2-6}
                    & Pop      & Rap      & Rock  &	R\&B  & Country\\\midrule
Average song rating &    0.432 &\bf 0.599 & 0.420 & 0.520 & 0.456 \\
Median song rating  &\bf 0.521 &    0.380 & 0.357 & 0.235 & 0.269 \\\bottomrule
\end{tabular}
\caption{Song rating per genre.} 
\label{song_rating_stats}
\end{table}

Looking at average and median \added{song} ratings in Table \ref{song_rating_stats}, we can observe two curious extremes -- rap having the highest average rating and pop with the highest median rating.
%Rap leading in the average, but this dominance not being translated into median, tells us that
So there must be some extremely high\added{ly} rated \added{rap} songs that pulled up the average.
\MP{Jednak jsem zde smazal/zakomentoval větu, která mi přišla zbytečným opakováním.
Jednak by to taky mohlo být tím, že jsou v popu písně s extrémně nízkým ratingem, na rozdíl od rapu, jak vlastně píšete v dalším odstavci.
Ono asi bude platit obojí.
To by mělo být vidět při vykreslení celé distribuce ratingů do grafu (třeba violin plot with whiskers, s kvartily, s vyznačenými outliers).
Asi to nechte, jak to je.
}
Although we did not predict this result, it shows that some artists probably took the importance of rhyme in rap very seriously and elaborately incorporated it densely into their lyrics.

\added{The} highest median in pop shows that many pop songs are filled with more rhymes\replaced{, which}{ what} can be explained by their strong tendency to be memorable. However, it seems that there are some low extremes that pulled the average rating down.

\chapter{Visualization}\label{visualization}
To make the results more approachable for a common user, it is always better to visualize them in some way. Therefore, to demonstrate our detector's capabilities, we created a website that visualizes rhymes and their quality, shows statistics, and allows user\added[comment={nebo člen}]{s} to experiment with the parameters. This way, it can be used by anyone without any programming background.

\section{Input}
The input page consists of a text-box for song lyrics, a card with parameters, and an \textit{Analyze} button, as seen in Figure \ref{web-form}.

The text-box expects text input of song lyrics, separated into verses with newlines such that rhymes are at the end of the line. Once text is entered, \textit{Analyze} button will be enabled.

For analysis, the default parameters are pre-filled, but user\added{s} can choose to change them. Selecting the checkbox \textit{Perfect rhymes only} will trigger the detector to only detect perfect rhymes.
\replaced{The \textit{Window} size specifies the maximum number of lines between rhyme fellows (window=1 means checking the previous line only).}{Changing the size of the window will affect how many lines apart can rhyme be.}
Smaller window is better for creating rhyme schemes, while longer window (e.g. equal to \added{the} song length) will give better overview of rhyme repetition throughout the entire song and give a more interesting matrix visualization. \textit{Rhyme threshold} parameter sets the minimal rhyme rating -- rhymes with lower rating will be discarded.

Pressing the \textit{Analyze} button will start the analysis and minimize the input page. For the duration of rhyme detection, \replaced{a loader box}{loader} is shown to inform the user\added{s} their request is being processed. When the back-end returns the results, \replaced{an \textit{Analyze}}{analyze}\todo{spíš bych tu page nazval Analysis či Results, ale na předělání screenshotů asi není čas, tak to nechte.} page is expanded to show the visualizations. If desired, user\added{s} can expand \added{the} input page, edit the input, and re-submit for analysis.

\begin{figure}[h]\centering
	\includegraphics[scale=0.3]{../img/web-empty-form.png}
	\caption{Website's form for entering the lyrics and setting the parameters.}
	\label{web-form}
\end{figure}

\section{Visualization of the results}
\added{The} visualization\todo{nebo The \textit{Analyze} page} page contains lyrics with scheme, matrix visualization of rhymes, and short statistics. It is primarily designed for songs of short or moderate length, longer lyrics may not fit on the screen with the analysis side-by-side, and will have to be rearranged in a column, which makes the results less comfortable to read.

\subsection{Lyrics and statistics}\label{sec-lyrics_and_stats}
Lyrics with their assigned scheme letters and line number are shown on the left, as we can see in Figure \ref{web-analysis_window5}. Rhyming lines are highlighted, each with a color corresponding to its rhyme type. For the sub-types of perfect rhyme, we selected similar colors to indicate that they are more closely related -- namely red for \textit{masculine}, orange for \textit{feminine}, and yellow for \textit{dactylic} rhymes. \textit{Imperfect} rhymes are highlighted in blue and \textit{forced} in green color. When \added{the} user hovers over a rhyming line, this line and all lines rhyming with it are highlighted.

Statistics \replaced{are}{is} shown on the right under the matrix. \replaced{They contain the}{It contains} song rating and percentages of different rhyme types in the song.

\begin{figure}[h]\centering
	\includegraphics[scale=0.25]{../img/love_game_lady_gaga.png}
	\caption[Screenshot from an analysis with the default window size.]{Screenshot from \replaced{an analysis with the}{analysis with} default window size. Example from \textit{Love Game} by Lady Gaga.}
	\label{web-analysis_window5}
\end{figure}

\subsection{Matrix}\label{visualization-matrix}
To make \replaced{the visualization more creative}{a creative visualization}, we took inspiration from Colin Morris.\footnote{\url{https://github.com/colinmorris}} He came up with an idea to represent the repetitiveness of lyrics by self-similarity matrix and he demonstrated it in his project SongSim.\footnote{\url{https://colinmorris.github.io/SongSim/\#/}} In his matrix, there is one row and one column for each word of the song. For each cell, if the word in given row and column are identical, the cell is colored, otherwise it stays white (Figure \ref{songsim}).

\begin{figure}[h]\centering
	\includegraphics[scale=0.2]{../img/songsim.png}
	\caption{Screenshot of Collin's SongSim visualization of song's repetitiveness.}
	\label{songsim}
\end{figure}

Instead of words, in our matrix, we compare \replaced{lines and highlight rhymes}{rhymes}. For rows and columns\added{,} we use rhyme scheme letters for \added{the} corresponding line, and when they agree, \added{the} matrix cell will receive the color of this rhyme's type, as described in Section \ref{sec-lyrics_and_stats} (Figure \ref{web-analysis_window5}). For comparison with \added{the} default window, in Figure \ref{web-analysis_window_all}, we include a screenshot with \added{a} longer window to better demonstrate the matrix.


\begin{figure}[!h]\centering
		\includegraphics[scale=0.24]{../img/europes-skies.png}
	\caption[Screenshot from an analysis with a window size matching the song length.]{Screenshot from an analysis with a window size matching the song length. Example from \textit{Europe's Skies} by Alexander Rybak.}
	\label{web-analysis_window_all}
\end{figure}

So far, \replaced{users have}{user has} been given a large-picture overview of \added{the} entire song. To explore the detail\added{s}, user\added{s} can view rhyme's properties by hovering over the particular matrix cell. \replaced{An pop-over box}{Popover} will display more details as shown in Figure \ref{web-popover}. \textit{Rhyming phonemes} for both rhyming lines display only phonemes participating in \added{the} rhyme -- meaning from \added{the} last stressed phoneme (or where the stress was moved) onward. Lines\deleted{,} that correspond to this rhyme, will also be highlighted in the text on the left. 

\begin{figure}[h]\centering
		\includegraphics[scale=0.45]{../img/popover-detail.png}
	\caption{Detail of \replaced{a pop-over box for a given}{popover over one} matrix tile.}
	\label{web-popover}
\end{figure}


\section{Technologies}
For the front-end of \added{the} web page\added{,} we used a TypeScript-based open-source web application framework \deleted{-- }Angular (\cite{angular}). As \added{a} design library, we used Bootstrap,\footnote{\url{https://getbootstrap.com/}} and ported version
\MP{Buď and a ported version, pokud to portoval někdo jiný.
Nebo and ported a version, pokud jste to portovala Vy.}
of standard Bootstrap's components to Angular -- \textit{ngx-bootstrap}.\footnote{\url{https://valor-software.com/ngx-bootstrap/}} \added{The} back-end simple REST API is written in Python using \added{a} micro-web framework \textit{Flask}.\footnote{\url{https://flask.palletsprojects.com/en/2.0.x/}} It calls our detector\deleted{,} in a classic variant without any modifications, as it was used for the evaluation in Chapter \ref{evaluation}. 

We host it at a public url \url{https://rhyme-detector.brezinovi.sk/} from our personal computer, so some short-term unavailability is possible. In case of any difficulties, please do not hesitate to contact us at \textit{patricia@brezinovi.sk}.

\chapter{Generation}\label{generation}
Writing song lyrics is a task that requires creativity. This is one of a few areas where AI has not become better than humans yet. While it is generally capable of keeping similar structure and topic, it still fails in using rhymes and other literary devices.

As described in Section \ref{generation_tools}, the current state-of-the-art tool for generation is GPT-3 \citep{brown2020gpt3}. We have sent several requests for their API but have not received a reply to this day.

The second-best option was its earlier version GPT-2 (\cite{radford2019gpt2}). It offers pre-trained models of sizes small, medium, large, and XL that can be further fine-tuned to the specific task. Since training models of such sizes has very high memory requirements, we would only be able to fine-tune the small model on available school clusters. 

However, our colleagues from ÚFAL at the Charles University agreed to let us generate on GPT-2 XL they are using in their ongoing research project THEaiTRE (\cite{rosa2021theaitre}). It is a vanilla model, meaning that it has not been fine-tuned to any specific task and we do not have resources to fine-tune it. But as their research confirms, it is not necessary because it adapts very well when given a relevant primer, i.e. a text snippet used as a start of the text to be generated. Knowing it has been trained on 8 million web pages, it is highly probable that it has already seen some lyrics data. 

Without fine-tuning, the quality of the result depends on the selection of the primer. To choose what works the best, we decided to compare the results using primers of different lengths. First, we tried one-line primers. As expected, it was not enough for GPT-2 to recognize it as poetry, and it continued with full paragraph as you can see in Figure \ref{fig_1line_primer}.

\begin{figure}[htb]\centering
	\minipage{0.9\textwidth}
	A chair is still a chair \\
	\color{blue}In most other aspects, I think the two camps in this debate can agree. We're in good company, for instance, as far as the law is concerned. When it comes to the ``should'' part of the debate, the fact that some states have no-fault insurance mandates, and others have them, is not a matter of opinion, or of opinion alone. \\
	\endminipage\hfill
	\caption{Result of generation (in blue) using a 1-line primer (in black).} \label{fig_1line_primer}
\end{figure} 

Subsequently, we iteratively increased the length of the primer exponentially. We conducted experiments with several song lyrics, but the results did not differ much, so we only show one example in Table \ref{generation_result}.

Except for the 4-line primer, it was able to successfully replicate the poem-like structure. It also used a lot of repetition, even when the original text did not contain any (e.g. the 2-line primer). We assume it must have already learned that repetition is very common for this kind of text structures. However, it was not successful in creating rhymes of its own.

Although longer primer seems to be a safer path to good results, they do not seem to be significantly better than the result from a 2-line primer. Occasionally, it appears to miss the clue and continue with regular text, as it did with the 4-line primer here.

%"url": ''http://genius.com/Luther-vandross-a-house-is-not-a-home-lyrics", "year": "1981-08-12", "lang": "en", "id": 13056, "word_count": 310,

\begin{table}[!h]
	\begin{tabular}{cc}
		\textbf{Original} &   \textbf{2-line primer} \\ \midrule
			\begin{minipage}[t]{0.47\textwidth}
		A chair is still a chair\\
		Even when there's no one sittin' there\\
		But a chair is not a house\\
		And a house is not a home\\
		When there's no one there to hold you tight\\
		And no one there you can kiss goodnight\\
		Woah girl
		\end{minipage}
		 & 
		\begin{minipage}[t]{0.47\textwidth}
		A chair is still a chair\\
		 Even when there's no one sittin' there\\
		 \color{blue} I still think there are people here\\
		 A chair is still a chair\\
		 Even when there's no one sittin' thereI still think there are people here\\
		 A chair is still a chair 
		 \end{minipage}\\[6pt]
	 \textbf{4-line primer} &   \textbf{8-line primer} \\ \midrule
	 \begin{minipage}[t]{0.47\textwidth}
	 	A chair is still a chair\\
	 	Even when there's no one sittin' there\\
	 	But a chair is not a house\\
	 	And a house is not a home\\
		\color{blue}
	 	So why do we make up the terms?\\
	 	It's not just us who insist on the word house, we use it in all sorts of other contexts too. For example, if you want to use a single word to refer to a group of people, you might say:\\
	 	If you want to invite everyone to an event, you might say:\\
	 	It's not hard to see why we seem to love using a term like 'house' to refer to a group of people – it's the easiest word for us to remember.
	 \end{minipage}
	 & 
	 \begin{minipage}[t]{0.47\textwidth}
	 A chair is still a chair\\
	 Even when there's no one sittin' there\\
	 But a chair is not a house\\
	 And a house is not a home\\
	 When there's no one there to hold you tight\\
	 And no one there you can kiss goodnight\\
	 Woah girl\\
	 \\
	 A room is a still a room\\
	 \color{blue} And a still a room is still a home\\
	 When there's no one there to hold you tight\\
	 And no one there you can kiss goodnight\\
	 I'm a little man on a little island\\
	 And a lonely little bird on a lonely little island\\	 
	 But a lonely little bird is not a castle\\
	 And a lonely little bird is not a home\\
	 And a lonely little bird is not a world\\
	 I'm a little man in a little house\\
	 And a lonely little bird in a lonely little house
	 \end{minipage}\\[6pt]
	\end{tabular}
	\caption{Beginning of ``A House Is Not A Home'' lyrics by Luther Vandross. Comparison with results generated using 2, 4, and 8-line primers.}
	\label{generation_result}
\end{table}

%Clouds Crash Lyrics by The matches, pop

Overall, GPT-2 succeeded in replicating the poetic form and structure. It creatively generated meaningful content that was close to human-written. A slight give-away is excessive repetition, but for an individual example, it could be mistaken for author's style.



\chapter*{Conclusion}\label{conclusion}
\addcontentsline{toc}{chapter}{Conclusion}
We started this thesis with a thorough research about rhyme and tools used for its detection, visualization, and generation. We explored our dataset and cleared it from impurities to improve our chances for a better analysis. After overcoming many obstacles of creating a rhyme detector, we evaluated it and performed an analysis over our entire dataset. At the end, we visualized the results and explored the generation using GPT-2.

Designing the detector was not easy, we rebuilt it several times as unpredictable exceptions came our way. The biggest difficulty was working with crowd-sourced data -- although we did what we could in the pre-processing phase, still in a dataset this large, there were words we had to deal with almost individually.

Another problem was the ambiguity of the resulting scheme. Often, there is no single correct scheme to be assigned. It is possible that different people would assign different schemes to the same song because someone would group all rhyming lines together under one letter, but another person may create separate groups by stanzas or other rules. Clearly, for evaluation we only have the gold scheme that was assigned by the annotator. We needed to compensate for this by adjusting the rhyme scheme in Section \ref{sec:scheme-adjustment} so that it is more reminiscent of common human annotation.

Although, in the comparison test, our detector did not outperform Rhyme Tagger, it was still a powerful detector and we believe it was a contribution to this research field. We tried new methods and approaches, and we were able to calculate statistics on almost half a million songs, which confirmed what we suspected about genre differences, but also gave evidence for new interesting findings. Our automated evaluation could, for some use-cases, replace human evaluators.

On top of that, we created an online web visualization that made this tool accessible for public. We implemented an innovative representation of rhymes using a self-similarity matrix.

Finally, we generated lyrics using GPT-2 and experimented with different primers, trying to achieve the best result. The generator was capable of replicating the form of the lyrics and even generating meaningful content. 

\section*{Future work}
In future, it would be worth to consider a more advanced data pre-processing or a cleaner dataset. Although cleaning such a big dataset has to be done automatically, every mistake can contribute to worse performance of both the detector and the generator. 

Alternatively, this pre-processing could be included in a more robust detector, that would take care of typing errors and automatically separate text into verses by rhymes.

For more comparisons with RhymeTagger, it could be interesting to train our detector on ChicagoRPC dataset (the same that RhymeTagger was trained on) and compare the results whether the poetry data help the detector learn rhymes better.

For more evaluation statistics, it could be interesting to design a metric that would evaluate the structure of the text (e.g. metre, rhythm, syllable count, and higher-level structure). Not only would this create a measure that would quantify how GPT-2 generated lyrics resemble human-written ones, but it would probably yield more interesting statistical differences between genres.

Another metric could be designed to evaluate repetitiveness in lyrics. Some repetition is common in lyrics but there is no implicit way to quantify how much is normal. This could also be used, in combination with other metrics, to automatically recognize machine-generated lyrics.

An interesting experiment would be to combine the detector and the generator to get better generated results. After generation of new lyrics, it could be evaluated and regenerated until the score reached a desired threshold.

Having access to GPT-3, we believe more impressing results in generation could be achieved.


%%% Bibliography
\include{bibliography}

%%% Figures used in the thesis (consider if this is needed)
\listoffigures

%%% Tables used in the thesis (consider if this is needed)
%%% In mathematical theses, it could be better to move the list of tables to the beginning of the thesis.
\listoftables

%%% Abbreviations used in the thesis, if any, including their explanation
%%% In mathematical theses, it could be better to move the list of abbreviations to the beginning of the thesis.
\clearpage
\printglossary[title=Glossary of literary and technical terms]\label{glossary-section}

%%% Attachments to the master thesis, if any. Each attachment must be
%%% referred to at least once from the text of the thesis. Attachments
%%% are numbered.
%%%
%%% The printed version should preferably contain attachments, which can be
%%% read (additional tables and charts, supplementary text, examples of
%%% program output, etc.). The electronic version is more suited for attachments
%%% which will likely be used in an electronic form rather than read (program
%%% source code, data files, interactive charts, etc.). Electronic attachments
%%% should be uploaded to SIS and optionally also included in the thesis on a~CD/DVD.
%%% Allowed file formats are specified in provision of the rector no. 72/2017.
\appendix
\chapter{Attachments}

\section{IPA and ARPAbet transcription table}
\label{ipa}
Following tables show the transcription between IPA and ARPAbet for consonants (Figure \ref{ipa-arpa-cons}) and vowels (Figure \ref{ipa-arpa-vowels}). The ARPAbet phoneme set used by CMUdict is shown, as described on their website\footnote{\url{http://www.speech.cs.cmu.edu/cgi-bin/cmudict}}. Note, that IPA diphthongs are not transcribed separately but as one two-character ARPAbet symbol.


\begin{table}[!ht]
	\centering
	\begin{tabular}{c c c} 
		ARPAbet & IPA & Example \\ [0.5ex] 
		\hline
		B & \textipa{b} & be \\
		CH & \textipa{tS} & cheese \\ 
		D & \textipa{d} & dee \\
		DH & \textipa{D} & thee \\
		F & \textipa{f} & fee \\
		G & \textipa{g} & green \\
		HH & \textipa{h} & he \\
		JH & \textipa{dZ} & gee \\
		K & \textipa{k} & key \\ 
		L & \textipa{l} & lee \\
		M & \textipa{m} & me \\
		N & \textipa{n} & knee \\
		NG & \textipa{N} & ping \\
		P & \textipa{p} & pee \\
		R & \textipa{r} & read \\
		S & \textipa{s} & sea \\
		SH & \textipa{S} & she \\
		T& \textipa{t} & tea \\
		TH & \textipa{T} & theta \\
		V & \textipa{v} & vee \\
		W & \textipa{w} & we \\
		Y & \textipa{j} & yield \\
		Z & \textipa{z} & zee \\
		ZH & \textipa{Z} & seizure \\
	\end{tabular}
	\caption{Consonant phonemes -- transcription between IPA and ARPAbet.}
	\label{ipa-arpa-cons}
\end{table}

\begin{table}[h!]
	\centering
	\begin{tabular}{c c c} 
		ARPAbet & IPA & Example \\ [0.5ex] 
		\hline
		AA  &\textipa{A}	&odd \\
		AE	&æ	&at	\\
		AH	&\textipa{2}	&hut	\\
		AO	&\textipa{O}	&ought	\\
		AW	&\textipa{aU}	&cow	\\
		AY	&\textipa{aI}	&hide	\\	
		EH	&\textipa{E}	&Ed	\\
		ER	&\textipa{3r}	&hurt	\\
		EY	&\textipa{eI}	&ate	\\
		IH	&\textipa{I}	&it	\\
		IY	&i				&eat	\\
		OW	&\textipa{oU}	&oat	\\
		OY	&\textipa{OI}	&toy	\\
		UH	&\textipa{U}	&hood	\\
		UW	&u	&two	\\
	\end{tabular}
	\caption{Vowel phonemes -- transcription between IPA and ARPAbet.}
	\label{ipa-arpa-vowels}
\end{table}



\openright
\end{document}
