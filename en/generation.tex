\chapter{Generation}\label{generation}
Writing song lyrics is a task that requires creativity. This is one of a few areas where AI has not become better than humans yet. While it is generally capable of keeping similar structure and topic, it still fails in using rhymes and other literary devices.

As described in Section \ref{generation_tools}, \added{the} current state-of-\added{the-}art tool for generation is GPT-3 \citep{brown2020gpt3}. We have sent several requests for their API but have not received a reply to this day.

The second-best option was its earlier version GPT-2 (\cite{radford2019gpt2}). It offers pre-trained models of sizes small, medium, large, and XL that can be further fine-tuned to the specific task. Since training models of such sizes has very high memory requirements, we would only be able to fine-tune the small model on available school clusters. 

However, our colleagues from ÚFAL at the Charles University agreed to let us generate on GPT-2 XL they are using in their ongoing research project THEaiTRE (\cite{rosa2021theaitre}). It is a vanilla model, meaning that it has not been fine-tuned to any specific task and we do not have resources to fine-tune it. But as their research confirms, it is not necessary because it adapts very well when given a relevant primer, i.e. a text snippet used as a start of the text to be generated. Knowing it has been trained on 8 million web pages, it is highly probable that it has already seen some lyrics data. 

Without fine-tuning, the quality of the result depends on the selection of the primer. To choose what works the best, we decided to compare the results using primers of different lengths. First, we tried one-line primers. As expected, it was not enough for GPT-2 to recognize it as poetry, and it continued with full paragraph as you can see in Figure \ref{fig_1line_primer}.

\begin{figure}[htb]\centering
	\minipage{0.9\textwidth}
	A chair is still a chair \\
	\color{blue}In most other aspects, I think the two camps in this debate can agree. We're in good company, for instance, as far as the law is concerned. When it comes to the ``should'' part of the debate, the fact that some states have no-fault insurance mandates, and others have them, is not a matter of opinion, or of opinion alone. \\
	\endminipage\hfill
	\caption{Result of generation (in blue) using \added{a} 1-line primer (in black).} \label{fig_1line_primer}
\end{figure} 

Subsequently, we iteratively increased the length of the primer exponentially. We conducted experiments with several song lyrics, but the results did not differ \deleted{very} much, so we \added{only} show one example in Table \ref{generation_result}.

Except for the 4-line primer, it was able to successfully replicate the poem-like structure. It also used a lot of repetition, even when the original text did not contain any (e.g. \added{the} 2-line primer). We assume\deleted{,} it must have already learned that repetition is very common for this \added{kind of} text structures. However, it was not successful in creating rhymes of its own.

Although longer primer seems to be a safer path to good result\added{s}, \replaced{they do}{it does} not seem to be significantly better than the result from \added{a} 2-line primer. Occasionally, it appears to miss the clue and continue with regular text, as it did with \added{the} 4-line primer here.

%"url": ''http://genius.com/Luther-vandross-a-house-is-not-a-home-lyrics", "year": "1981-08-12", "lang": "en", "id": 13056, "word_count": 310,

\begin{table}[!h]
	\begin{tabular}{cc}
		\textbf{Original} &   \textbf{2-line primer} \\ \midrule
			\begin{minipage}[t]{0.47\textwidth}
		A chair is still a chair\\
		Even when there's no one sittin' there\\
		But a chair is not a house\\
		And a house is not a home\\
		When there's no one there to hold you tight\\
		And no one there you can kiss goodnight\\
		Woah girl
		\end{minipage}
		 & 
		\begin{minipage}[t]{0.47\textwidth}
		A chair is still a chair\\
		 Even when there's no one sittin' there\\
		 \color{blue} I still think there are people here\\
		 A chair is still a chair\\
		 Even when there's no one sittin' thereI still think there are people here\\
		 A chair is still a chair 
		 \end{minipage}\\[6pt]
	 \textbf{4-line primer} &   \textbf{8-line primer} \\ \midrule
	 \begin{minipage}[t]{0.47\textwidth}
	 	A chair is still a chair\\
	 	Even when there's no one sittin' there\\
	 	But a chair is not a house\\
	 	And a house is not a home\\
		\color{blue}
	 	So why do we make up the terms?\\
	 	It's not just us who insist on the word house, we use it in all sorts of other contexts too. For example, if you want to use a single word to refer to a group of people, you might say:\\
	 	If you want to invite everyone to an event, you might say:\\
	 	It's not hard to see why we seem to love using a term like 'house' to refer to a group of people – it's the easiest word for us to remember.
	 \end{minipage}
	 & 
	 \begin{minipage}[t]{0.47\textwidth}
	 A chair is still a chair\\
	 Even when there's no one sittin' there\\
	 But a chair is not a house\\
	 And a house is not a home\\
	 When there's no one there to hold you tight\\
	 And no one there you can kiss goodnight\\
	 Woah girl\\
	 \\
	 A room is a still a room\\
	 \color{blue} And a still a room is still a home\\
	 When there's no one there to hold you tight\\
	 And no one there you can kiss goodnight\\
	 I'm a little man on a little island\\
	 And a lonely little bird on a lonely little island\\	 
	 But a lonely little bird is not a castle\\
	 And a lonely little bird is not a home\\
	 And a lonely little bird is not a world\\
	 I'm a little man in a little house\\
	 And a lonely little bird in a lonely little house
	 \end{minipage}\\[6pt]
	\end{tabular}
	\caption{Beginning of ``A House Is Not A Home'' lyrics by Luther Vandross. Comparison with results generated using 2, 4, and 8-line primers.}
	\label{generation_result}
\end{table}

%Clouds Crash Lyrics by The matches, pop

Overall, GPT-2 succeeded in replicating the poetic form and structure. It creatively generated meaningful content that was \deleted{very} close to human-written. A slight give-away is excessive repetition, but for an individual example, \added{it} could be mistaken for author's style.
\MP{Osobně bych GPT-2 u básní nehodnotil tak pozitivně,
ale snad je jasné, že je to spíš Váš názor (žádná exaktní evaluace).
Aspoň jsem ale smazal to ``very'' close.}
