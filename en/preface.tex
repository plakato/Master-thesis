\chapter*{Introduction}
\addcontentsline{toc}{chapter}{Introduction}

As artificial intelligence keeps catching up with humans, even despite numerous attempts, in artistic fields people still prefer human-made art.
For computers, it is hard to create art, and even harder to understand and analyze it.

A piece of art in everyday life of almost everyone is music. It is a complex form, where many aspects influence the audience, i.e. melody, rhythm, lyrics, performance, etc. Although we do realize they are interconnected and may affect each other, in this thesis, we will more deeply explore only one of these aspects -- song lyrics. 

We have a large crowd-sourced dataset of almost half a million song lyrics. At first, this sounded as a good base for learning a lyrics generator. However, as we explored rhymes and automatic analysis, we realized it is a much more interesting path to pursue. Every attempt at lyrics or poetry generation that we encountered used humans for their final evaluation. This proves, and \cite{greene2010automatic} agree, that automatic evaluation of poetry is hard.

Unfortunately, there was no sufficient rhyme detector that we could use for our case. In this thesis, we will dive more deeply into the problem and create one ourselves. It will give us the ability to analyze our dataset and draw interesting conclusions about the data. 

Additionally, we will create a web-page that demonstrates detector's capabilities and visualizes rhymes in an innovative way. With this tool, we hope to give artists, authors of poems and songs, or even amateurs a new way to explore their texts.

At the end, we will focus on lyrics generation and explore current state-of-the-art
pre-trained GPT-2 model and its capabilities in this field.

This work may include some literary or technical terms that the reader is not familiar with. For their definition, please see the \hyperref[glossary-section]{``Glossary of literary and technical terms''} section at the end of this thesis.


\section*{Outline}
In Chapter \ref{chap-related-work}, we will make we will explain the literary background such as rhyme, its types, and other literary devices. We will also describe approaches and review existing tools for rhyme detection, visualization, and lyrics generation. 

Chapter \ref{data} introduces data that we will be working with, their structure and statistics, and the steps we took to pre-process them.

The most complex part of this thesis is explained in Chapter \ref{chap-rhyme-analysis}, which specifies the details of how we perform rhyme detection in song lyrics.

Chapter \ref{evaluation} evaluates our rhyme detector and shows the statistics when we run it on our dataset.

How the output of our detector is brought to life by visualization is illustrated in Chapter \ref{visualization}.

In Chapter \ref{generation}, we describe and review the results of lyrics generation experiment.

Lastly, the results are summed up in Conclusion (Chapter \ref{conclusion}), including suggestions for future work.
